% !TEX TS-program = pdflatex
% !TEX encoding = UTF-8 Unicode

% This is a simple template for a LaTeX document using the "article" class.
% See "book", "report", "letter" for other types of document.

\documentclass[12pt]{book} % use larger type; default would be 10pt

\usepackage[utf8]{inputenc} % set input encoding (not needed with XeLaTeX)

%%% Examples of Article customizations
% These packages are optional, depending whether you want the features they provide.
% See the LaTeX Companion or other references for full information.

%%% PAGE DIMENSIONS
%\usepackage{geometry} % to change the page dimensions
\usepackage[papersize={5.5in,8.5in},tmargin=15mm,bmargin=25mm,lmargin=25mm,rmargin=15mm]{geometry}
%\geometry{letterpaper} % or letterpaper (US) or a5paper or....
% \geometry{margin=2in} % for example, change the margins to 2 inches all round
% \geometry{landscape} % set up the page for landscape
%   read geometry.pdf for detailed page layout information

\usepackage{graphicx} % support the \includegraphics command and options

% \usepackage[parfill]{parskip} % Activate to begin paragraphs with an empty line rather than an indent

%%% PACKAGES
\usepackage{booktabs} % for much better looking tables
\usepackage{array} % for better arrays (eg matrices) in maths
\usepackage{paralist} % very flexible & customisable lists (eg. enumerate/itemize, etc.)
\usepackage{verbatim} % adds environment for commenting out blocks of text & for better verbatim
\usepackage{subfig} % make it possible to include more than one captioned figure/table in a single float
% These packages are all incorporated in the memoir class to one degree or another...

%%% HEADERS & FOOTERS
\usepackage{fancyhdr} % This should be set AFTER setting up the page geometry
\pagestyle{fancy} % options: empty , plain , fancy
\renewcommand{\headrulewidth}{0pt} % customise the layout...
\lhead{}\chead{}\rhead{}
\lfoot{}\cfoot{\thepage}\rfoot{}

%%% SECTION TITLE APPEARANCE
\usepackage{sectsty}
\allsectionsfont{\sffamily\mdseries\upshape} % (See the fntguide.pdf for font help)

\usepackage{fancyhdr} % This should be set AFTER setting up the page geometry
\pagestyle{fancy} % options: empty , plain , fancy
\renewcommand{\headrulewidth}{0pt} % customise the layout...
\lhead{}\chead{}\rhead{}
\lfoot{}\cfoot{\thepage}\rfoot{}

% (This matches ConTeXt defaults)
\usepackage[spanish]{babel}

%%% ToC (table of contents) APPEARANCE
\usepackage[nottoc,notlof,notlot]{tocbibind} % Put the bibliography in the ToC
\usepackage[titles,subfigure]{tocloft} % Alter the style of the Table of Contents
\renewcommand{\cftsecfont}{\rmfamily\mdseries\upshape}
\renewcommand{\cftsecpagefont}{\rmfamily\mdseries\upshape} % No bold!

%%% END Article customizations

%%% The "real" document content comes below...

\title{Consejos financieros para casa}
\author{Efraín Serna Gracia}
%\date{} % Activate to display a given date or no date (if empty),
         % otherwise the current date is printed 

\begin{document}
\maketitle
\tableofcontents
\chapter{Introducción}
¿Tienes problemas financieros? ¿Sientes que no avanzas? ¿Tu familia o tú vives en un constante estado de tensión porque el dinero no alcanza? Conozco perfectamente estas sensaciones, más aún, no me gusta mostrar cómo me afecta esto y me muestro siempre feliz.\\

Con base en mi experiencia personal te comparto un breve conjunto de consejos que pueden ayudarte a salir del agujero. Son consejos basados en algunos libros que he leído, en el sentido común y en los errores que he cometido, sobretodo esto último. ``De los errores se aprende, ¿del éxito? No tanto''.\\

Esperando que el contenido de este pequeño libro sea de utilidad, disfruta su lectura, y más importante: pon en práctica estos consejos.

\chapter{Cómo utilizar estos apuntes}
Cada consejo es independiente aunque conectados funcionan mejor. Los consejos son para salir del hoyo financiero y lograr una estabilidad y tranquilidad independientemente de la forma en que tengas tus ingresos.

Subraya, marca, resalta las partes que consideres más importantes para tu situación particular.

Comenta y agrega lo que consideres mejor en cada parte.

\chapter[Los huevos en la canasta]{Problema 1: los huevos en la canasta}
Si me piden un consejo para mejorar las finanzas en casa, creo que el primero y más básico sería este: \emph{NO poner todos los huevos en la canasta}.\\

¿Qué quiero decir con esto? Pues muy fácil. A la fecha de hoy, en casa hemos tenido un problema, y es que siempre estamos corriendo al día, no importa lo que hagamos siempre estamos con deudas, con números rojos, pidiendo prestado, empeñando, tronándonos los dedos para llegar a fin de mes. Tampoco importa cuál sea nuestro ingreso, siempre estamos igual.\\

¿Les ha pasado? , ¿les suena conocido? La verdad es que es algo estresante, es una carga constante, no deja dormir tranquilamente, mantiene un ambiente tenso en casa y cualquier comentario que aluda a realizar gasto o pago causa sensación de frustración. Lo he experimentado en carne propia. Es más, hay veces que dicen ``me preocupa tal deuda'' que sé es pequeña en comparación con las que he tenido.\\

Mi consejo, nuevamente, es: \emph{no poner todos los huevos en la canasta}, hagan un hábito de separar una pequeña cantidad tan pronto se recibe dinero, 1 \%, 5 \%, 10 \%, lo que sea es bueno, y colóquenlo en un lugar del que no lo puedan sacar. Se puede pedir ayuda de álguien de confianza, de álguien que sabemos mantiene una tranquilidad financiera y un crecimiento económico porque esa persona no tocará el dinero que le confiemos, por ejemplo, los padres. Al principio que sea una cantidad que puedan permitirse, y con el tiempo pueden incrementar esa cantidad.\\

El objetivo es aprender a vivir con menos de lo que entra a la casa. ¿Por qué? porque nos hemos acostumbrado a gastar más de lo que recibimos, y es justamente eso lo que nos ha llevado al estado actual; ahora lo que buscamos es hacer lo opuesto, gastar menos de lo que ganamos. Al principio será difícil, de aquí lo importante es mantenernos de común acuerdo todo el tiempo.\\

Un ejemplo es: todos los días, al finalizar el día, las monedas de baja denominación júntenlas en un frasco y al final de la semana ese monto lo entregan a quien hayan destinado como ``caja fuerte'', y lleven un registro conjunto con la persona de cuánto se entrega y en qué fecha. Otra opción es generar una cuenta bancaria y entregar la tarjeta a su persona ``caja fuerte'' para que ustedes no tengan acceso sino hasta el final del plazo que se hayan propuesto.\\

Esto permite concretar pequeños deseos o proyectos familiares cada cierto tiempo pagando en efectivo y ayudando a mejorar el estado de ánimo de casa. Además se fortalece la voluntad de la familia.\\

\chapter[La ruta al Dorado]{Problema 2: La ruta al Dorado}
Si me preguntan por  algo para hacer rendir nuestro dinero, con base en mi experiencia diré: \emph{Hacer un presupuesto y respetarlo}.\\

¿Un presupuesto? ¡Si somos una familia, no una empresa o un gobierno! Pues sí, pero si no hacen un presupuesto, entonces no van a saber hacia dónde deben destinar sus recursos, tampoco sabrán qué tanto están destinando a cada aspecto. Un viejo proverbio dice ``Ningun viento es favorable para el marinero que no sabe a dónde va''.\\

El presupuesto, entre más detallado sea es mejor.\\

Comiencen por hacer una lista de todos los gastos frecuentes que han de realizar para la subsistencia de la casa, esos gastos serán más o menos fijos y son de obligación a cubrir. Comida, servicios, transporte, cuotas de escuela... todo eso deberá entrar ahí, y asígnenle la cantidad que han estado destinando. Si descubren que no pueden precisar la cantidad destinada a la comida (como nos pasa a nostros), entonces sabrán que ahí hay algo a qué prestar atención. También deberán poner en qué día del mes realizan esos pagos fijos.\\

A continuación hagan una lista de todos los compromisos financieros que tengan con otros, como por ejemplo: créditos, empeños, préstamos solicitados. De igual forma anoten el monto que están cubriendo por ellos y en qué días lo hacen. Esto les mostrará cuánto de su ingreso está siendo utilizado para llenar bolsillos ajenos (en caso de créditos bancarios y/o empeños, especialmente estos últimos) o a quienes no deben defraudar la confianza.\\

A lo largo del mes, por ejemplo, anoten si efectivamente ya fue cubierto el pago o qué tanto se destinó para ello.\\

De ser posible, tan pronto intre dinero a casa han de realizar el pago de estos compromisos. Es más sencillo apretarse el cinturón un poquito, racionar un poquito la comida e incluso pedir ayuda por una cantidad pequeña y devolverla a la brevedad que por no cubrir esos compromisos se caiga en un agujero financiero mayor (nuevo empeño, nuevo préstamo mediano a grande, cobro de intereses).\\

El objetivo es salir de los compromisos más grandes lo más pronto posible con la menor cantidad de recursos. Normalmente, al ser grandes esos compromisos es tardado salir de ellos pero con constancia se logra.

\chapter[El barril sin fondo]{Problema 3: El barril sin fondo}
Como llenar de agua un agujero en la playa, ¿Lo han intentado? ¿Qué pasa? Efectivamente, tan pronto poenen el agua esta se filtra dejando el hueco vacío. ¿Qué hacen para evitar que el agua escape y poder hacer una pequeña alberca? ¿No a caso ponen un plástico en el interior a fin de que cubra toda la arena y no se filtre más el agua?\\

Lo mismo pasa con nuestro dinero cuando pedimos un préstamo en una casas de empeño, si bien el pago por mantener el empeño vigente es pequeño en comparación con el dinero que nos prestan, lo cierto es que al paso del tiempo este dinero llega a equiparar o sobrepasar el préstamo original. Nos ha pasado en casa, hemos tenido objetos en empeño por mucho tiempo, y si hacemos suma pues el total excede por mucho lo prestado originalmente. Por ejemplo, si nos prestan 1000 acordando pagos de 100 cada mes para evitar la venta de nuestros objetos de valor y un pago de 1500 por recuperarlo, entonces luego de un año y tres meses habremos igualado el monto para recuperarlo pero sin poder hacerlo, ya que habremos pagado cada mes los 100 para mantenerlo vigente. A eso hay que agregar los 1500 para recuperarlo, resultando en un pago igual al doble de lo acordado originalmente para recuperarlo.\\

Adicionalmente, el tener que hacer el pago de manera constante sin obtener un beneficio por ello nos supone una carga moral extra que nos causa frustración.\\

Mi consejo es: \emph{Tapen las fugas de dinero, no lo echen en un barril sin fondo}. Si siguen los dos consejos anteriores encontrarán la forma de reunir el capital para recuperar esos bienes a la brevedad, lo que les ayuda a tener más tranquilidad, ya que el dinero que sería destinado después para pagar por mantener vigentes los empeños se puede utilizar para cubrir otras cosas o para el ahorro. A fin de cuentas era una cantidad con la que ya no contaban para cubrir sus necesiades.

\chapter[Plaga de hormigas]{Problema 4: Plaga de hormigas}
¡Vamos a comer tacos!, ¡Se me antojó un refresco!, ¡Quiero comprar unas plantas!, ¿Y si nos vamos a comer a un restaurant?... ¿Les suena familiar esto?\\

Muchas veces nos damos pequeños gustos y antojos, y está bien, merecemos pequeñas recomensas, el problema radica en que si se está en una situación financiera difícil estos pequeños antojos pueden marcar diferencia. Por ejemplo, y al momento de escribir esto: si compramos una bolsa de papas fritas en la calle y un refresco de 600 ml, ahí nos estaremos gastando 25 pesos que equivalen a (dependiendo de la marca) uno a 2 litros de leche de caja o hasta tres litros de leche del lechero, si eso lo compramos uno para cada integrante de la familia, en una familia de cuatro ya se fueron 100 pesos que equivalen a 1/4 de jamón, 1 paquete de pan de caja, un paquete de queso, o también alcanza para comprar verduras y pollo para un caldo de pollo.\\

Los ejemplos anteriores, si los proyectamos a la frecuencia con que nos damos \emph{recompensas}, veremos que se reune una cantidad importante a fin de mes, cantidad que podemos utilizar para cubrir otras necesidades y ayudarnos a salir de compromisos. Siempre teniendo presente que durante el tiempo de dificultad deberíamos hacer esto y saliendo podemos volver a recompensarnos, siempre con medida.\\

\chapter[Repartiendo al venado]{Problema 5: Repartiendo al venado}
``¿Compramos esto para Fulanito?'', ``A Perenganito le falta esto'', ``Sutanita no tiene para tal cosa'', ``Hay que comprarle esto a Menganita''. ¿Cuántas veces hemos tenido el ``buen corazón'' de ayudar a los demás cuando en el momento tenemos? ¿Se han encontrado en estas situaciones? No son los hijos sino gente cercana, especialmente familiares. Les ayudamos porque carecen de los recursos para cubrir sus necesidades y en ese momento a nosotros nos entra el dinero fruto del trabajo. Es muy loable, muy noble prestar ayuda pero... ¿A costa de que luego nosotros sigamos psando penurias? ¿Cuántas veces por ayudar a alguien luego tenemos la sensación ``de no haber usado ese dinero así tendría para pagar esto otro''...? ¿Frustración como consecuencia de ayudar?\\
Es verdad que la dádiva entregada con caridad nos es devuelta con creces en otro momento,  pero también hay que ser prudentes.

\chapter{Del antiguo sobre el dinero...}
\begin{description}
\item[Eclesiástico, 18:33] ``No te empobrezcas festejándote con dinero prestado cuando no tienes nada en tu bolsillo''
\item[Eclesiástico, 29:2-6] ``Presta a tu prójimo si lo necesita; por tu parte, págale a tu prójimo en el plazo acordado.
Manten tu palabra, sé leal con tu prójimo, y en cualquier momento tendrás lo que necesitas.
Muchos consideran el préstamo que se les hace como un regalo, así ponen en apuros a los que los ayudaron.
Mientras no hayan recibido, besarán las manos del prójimo, se harán los humildes pensando en lo que él tiene; pero en el momento de la devolución le piden una prórroga, o le pagan con palabras hirientes, o se quejan de la mala situación.
Aunque puedan reempolsarte, el acreedor tendrá  suerte si recupera la mitad. En caso contrario se habrán  apoderado de su dinero y no habrá ganado más que un enemigo; éste le pagará con maldiciones e insultos, con desprecios en vez de agradecimientos.''
\item[Eclesiástico, 29:20]``Acude en ayuda de tu prójimo en la medida de tus posibilidades, pero trata de no caerte''
\end{description}
\end{document}

% cama de lechuga
% zanahoria, jícama rayadas
% rodajas de jícama
% jugo de naranja como aderezo

