\documentclass[landscape,12pt,twocolumn]{article} % use larger type; default would be 10pt

\usepackage[utf8]{inputenc} % set input encoding (not needed with XeLaTeX)

%%% PAGE DIMENSIONS
\usepackage{geometry} % to change the page dimensions
\geometry{letterpaper} % or letterpaper (US) or a5paper or....
\usepackage{graphicx} % support the \includegraphics command and options

%%% PACKAGES
\usepackage{booktabs} % for much better looking tables
\usepackage{array} % for better arrays (eg matrices) in maths
\usepackage{paralist} % very flexible & customisable lists (eg. enumerate/itemize, etc.)
\usepackage{verbatim} % adds environment for commenting out blocks of text & for better verbatim
\usepackage{subfig} % make it possible to include more than one captioned figure/table in a single float

%%% HEADERS & FOOTERS
\usepackage{fancyhdr} % This should be set AFTER setting up the page geometry
\pagestyle{fancy} % options: empty , plain , fancy
\renewcommand{\headrulewidth}{0pt} % customise the layout...
\lhead{}\chead{}\rhead{}
\lfoot{}\cfoot{\thepage}\rfoot{}

%%% SECTION TITLE APPEARANCE
\usepackage{sectsty}
\allsectionsfont{\sffamily\mdseries\upshape} % (See the fntguide.pdf for font help)

%%% ToC (table of contents) APPEARANCE
\usepackage[nottoc,notlof,notlot]{tocbibind} % Put the bibliography in the ToC
\usepackage[titles,subfigure]{tocloft} % Alter the style of the Table of Contents
\renewcommand{\cftsecfont}{\rmfamily\mdseries\upshape}
\renewcommand{\cftsecpagefont}{\rmfamily\mdseries\upshape} % No bold!

\title{Aprendiendo latín}
\author{Efraín Serna Gracia}
%\date{} % Activate to display a given date or no date (if empty),
         % otherwise the current date is printed 

\begin{document}
\maketitle

\section*{Introducción}
Estoy aprendiendo latín y tomo varias fuentes, entre páginas donde comentan frases de uso corriente como de cursos más formales, un grupo de aprendizaje de latón en Facebook y un libro que en el mismo grupo utilizan como material de curso.\\

Específicamente: \emph{LINGVA LATINA PER SE ILLVSTRATA - Pars I - Familia Romana}, también el libro de Latín Elemental, segunda edición, de René Muñoz De La Fuente
\section{Comenzando}
En el latín se tiene seis casos que se relacionan con diferentes preposiciones y elementos de la oración, a saber:
\begin{description}
\item[Nominativo] \{el/la/lo y un/una\} Refiere al nombre de las cosas, el sujeto en una oración simple. Por ejemplo: \emph{la rosa, el águila, una sala, un camino}.
\item[Genitivo] \{de\} Para indicar posesión, materia o cualidad. Por ejemplo: camino \emph{del agua}, sala \emph{de la clase}, libro \emph{de sabiduría}, casa \emph{de alegría}.
\item[Dativo] \{a/para\}Indica sobre quien recae  de manera indirecta el efecto de lo expresado. Por ejemplo: rosas \emph{para la iglesia}.
\item[Acusativo] \{a\}. Por ejemplo: \emph{}
\item[Vocativo]\{¡oh!\}. Por ejemplo: \emph{}
\item[Ablativo]\{con/en/por\}. Por ejemplo: \emph{}
\end{description}
\subsection{Casos de la primera declinación}
Las siguientes terminaciones se utilizan para cada uno de los casos especificados.
\begin{description}
\item[-a] Nominativo singular ($N_S$), Vocativo singular ($V_S$), Ablativo Singular ($Ab_S$).
\item[-ae] Genitivo Singular ($G_S$), Dativo Singular ($D_S$), Nominativo Plural ($N_P$), Vocativo Plural ($V_P$).
\item[-am] Acusativo Singular ($Ac_S$).
\item[-arum] Genitivo Plural ($G_P$).
\item[-is] Dativo Plural ($D_P$), Ablativo Plural ($Ab_P$).
\item[-as] Acusativo Plural ($Ac_P$).
\end{description}

\subsection{Vocabulario}
\begin{tabular}{ll}
Ecclesia (\emph{Iglesia}) & Columba (\emph{Paloma})\\
Lácrima (\emph{Lágrima}) & Via (\emph{Camino})\\
Schola (\emph{Clase}) & Aula (\emph{Sala})\\
Porta (\emph{Puerta}) & Sapientia (\emph{Sabiduría})\\
Silva (\emph{Selva}) & Áquila (\emph{Águila})\\
Aqua (\emph{Agua}) & Laetitia (\emph{Alegría})\\
Vita (\emph{Vida}) & Puella (\emph{Niña})\\
Tábula (\emph{Pizarrón}) & Mensa (\emph{Mesa})\\
Fenéstra (\emph{Ventana}) & Sella (\emph{Silla}).
\end{tabular}

\section{Algo para el día a día}
\begin{tabular}{rl}
Salve & Hola (a una persona)\\
Salvete & Hola (a varias personas)\\
Vale & Adiós (a una persona)\\
Valete & Adiós (a varias personas)\\
Ut vales? & ¿Cómo estás?\\
Optime valeo & Estoy muy bien\\
Bene valeo & Estoy bien\\
Satis bene valeo & Estoy más o menos bien\\
Non ita bene valeo & No estoy muy bien\\
Pessime valeo & Estoy muy mal\\
Gratias ago & Gracias
\end{tabular}

\section{Familia Romana}
Interesante porque aunque todo está en latín, comienza con oraciones simples como \emph{Italia in Europa est} que se traduce como \emph{Italia en Europa está} e \emph{Italia et Graecia in Europa sunt} como \emph{Italia y Grecia en Europa están} o \emph{Italia est in Europa} que se traduce como \emph{Italia está en Europa}.\\

Su correspondiente negación con ejemplos como \emph{Aegyptus in Europa non est} que significa (literalmente) \emph{Egipto en Europa no está}, en plural sería \emph{Graecia et Roma non in Hispania sunt}.\\

Para preguntar es \emph{Estne Roma in Gallia?} que significa \emph{¿Roma está en Galia?}, o \emph{Ubi est Roma?} para decir \emph{¿Dónde está Roma?}.\\

\tableofcontents

\end{document}