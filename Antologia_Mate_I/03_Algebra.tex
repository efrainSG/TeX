\chapter{Álgebra}
\section{Conceptos básicos}
Al elemento más simple en álgebra se le conoce como término. Este consta de:
\begin{description}
\item[Signo.] Puede ser positivo o negativo, colocado a la izquierda. Es frecuente que el signo positivo no se dibuje, salvo para separar términos en una expresión.
\item[Coeficiente.] Es un número real colocado a la izquierda de las literales. Puede ser expresado como número entero, fraccionario o con decimales. Hay algunos símbolos que los pueden representar, por ejemplo: $e$ o $\pi$.
\item[Literal.] Una o más, en forma consecutiva, sin signos que las separe, colocadas a la derecha del coeficiente.
\item[Exponente.]Término que está colocado del lado derecho y arriba del coeficiente o de las literales.
\end{description}
Ejemplo de términos:
\begin{center}
$$
\begin{array}{ccc}
2a^2b & -13bd^3 & -1\\
a^3b^4 & -13^{ax}b^{2x} & a^3b^4
\end{array}
$$
\end{center}
\section{Lenguaje matemático}
Es un lenguaje que permite construir expresiones matemáticas a partir de un lenguaje natural y comprenderlas.
$$
\begin{array}{|ll|ll|}
\hline
\text{Un número.} & a & \text{El doble de un número.} & 2a \\
& & & \\
\hline
\text{Dos números distintos.} & a, b & \text{El doble de la suma de} & 2(a+b) \\
& &\text{dos números.}& \\
\hline
\text{La suma de dos números.} & a + b & \text{El doble de la resta de} & 2(a-b) \\
& &\text{dos números.}& \\
\hline
& & \text{El producto de dos} & \\
\text{La resta de dos números.} & a - b &\text{ números más el producto} & ab+cd \\
& &\text{de otros dos números.} & \\
\hline
\text{El producto de} & & \text{El producto de dos números} & \\
\text{dos números.}& ab, (a)(b) &\text{menos el producto de otros} & ab-cd \\
& &\text{dos números.} & \\
\hline
 & & \text{El cociente de dos números} & \\
\text{La mitad de un número.} & \dfrac{a}{2} &\text{más el cociente de otros dos}& \dfrac{a}{b}+\dfrac{c}{d} \\
& &\text{números.}& \\
\hline
\text{El cociente de dos} & & \text{El cociente de dos números} & \\
\text{números.}& \dfrac{a}{b}, a \div b &\text{menos el cociente de otros}& \dfrac{a}{b}-\dfrac{c}{d} \\
& &\text{dos números.} & \\
\hline
\text{El cuadrado de un} & & \text{La suma de dos números} & \\
\text{número.}& a^2 &\text{por la suma de otros dos} & (a+b)(c+d) \\
& &\text{números.} & \\
\hline
\text{La semi suma de dos} & & \text{La resta de dos números} & \\
\text{números.}& \dfrac{a+b}{2} &\text{por la resta de otros dos}& (a-b)(c-d) \\
& &\text{números.} & \\
\hline
\text{La semi resta de dos} & & & \\
\text{números.}& \dfrac{a-b}{2} & & \\
& & & \\
\hline
\end{array}
$$
Por ejemplo:
\begin{enumerate}
\item En una bolsa hay 5 canicas rojas, 4 canicas azules y 8 canicas blancas:\\
Las canincas se van a representar con letras, a saber:
\begin{itemize}
\item canicas rojas $\Rightarrow r$
\item canicas azules $\Rightarrow a$
\item canicas blancas $\Rightarrow b$
\end{itemize}
La expresión que indica el total de canicas de la bolsa será: $5r+4a+8b$
\item Para calcular la superficie de un terreno rectangular se multiplica lo que mide de largo por lo que mide de ancho, es decir \emph{el producto de dos números}, lo que resulta en: $S=bh$.
\item La velocidad a la que un auto se mueve se obtiene al dividir la distancia que ha recorrido entre el tiempo empleado en recorrerla (el cociente de dos números), lo que resulta en $velocidad=\dfrac{distancia}{tiempo}$ o en $v=\dfrac{d}{t}$.
\item Para calcular la aceleración se realiza la división de la resta de la velocidad final menos la velocidad inicial entre el tiempo empleado: $aceleracion=\dfrac{velocidad_{final} - velocidad_{inicial}}{tiempo}$ o lo que es lo mismo $a=\dfrac{v_f - v_i}{t}$
\end{enumerate}
\subsection{Ejercicios}
\begin{enumerate}
\item El semiproducto de dos números. ¿Dónde se ha visto anteriormente la expresión resultante?
\item El semiproducto de: la suma de dos números multiplicada por un tercer número. ¿Dónde se ha visto anteriormente la expresión resultante?
\item Como en el ejemplo de las canicas, usando \emph{u} para ``unos'', \emph{d} para ``dos'' y \emph{t} para ``tres'', ¿cuál es la expresión para indicar cuántos ``unos'', ``dos'' y ``tres'' hay en los números entre el $1$ y el $50$?
\item Una habitación mide 10 pies con 7 pulgadas de largo y 8 pies con 8 pulgadas de ancho. ¿Cuál es la expresión que indica el cálculo de su superficie?
\end{enumerate}
\section{Sumas y restas}
Para realizar sumas y restas algebráicas sólamente se pueden sumar y restar cantidades que tengan las mismas literales con los mismos exponentes. A continuación se ejemplifica el proceso:\\

$
\begin{array}{l|{l}}
3a+4b-5a+2b+7a-10b= & \\
3a-5a+7a+4b+2b-10b= &  \text{Se ordena agrupando por letras.}\\
5a-4b & \text{Se suman/restan las cantidades.}\\
 & \text{Si quedan términos con letras diferentes,}\\
 & \text{se dejan indicados así.}
\end{array}
$\\

\subsection{Ejercicios}
$$
\begin{array}{l}
2a+3b =\\
4b-5c+3b+8c-12b=\\
-4x+7y-18x+5y-9x+17y-5x+8y=\\
4x^2y+5x^3-8x^2y+3y-10x+15y^3+5x^2y-8x+19y-20x^2y=
\end{array}
$$
\section{Sumas y restas con signos de agrupación}
Los signos de agrupación sirven para modificar el orden o \emph{jerarquía} de las operaciones. Para resolverlas se hace desde el par agrupador que está más adentro.\\
Los signos de agrupación son:\\
\begin{center}
$$
\begin{array}{ccc}
[a+b+c] & \{a+b+c\} & (a+b+c)\\
\langle a+b+c \rangle & |a+b+c| & \|a+b+c\|\\
\overbrace{a+b+c} & \overline{a+b+c} & \underbrace{a+b+c}\\
 & \underline{a+b+c} & \\
\end{array}
$$
\end{center}

Hay que tener dos consideraciones. Los signos $+$ y $-$ separan los términos agrupados, para estos casos tenemos:
\begin{center}
$
\begin{array}{ccl}
-(2x-3) & -(-4y+2) & \text{Cuando se tienen signos de agrupación precedidos por}\\
-2x+3 & 4y-2 & \text{signo negativo. En este caso se quitan los signos de}\\
& & \text{agrupación y el signo negativo, y se invierten los signos}\\
 & & \text{de la expresión dentro de los signos de agrupación.}\\
\hline
+(2x-5y) & +(-4y+3z) & \text{Cuando se tienen signos de agrupación precedidos por}\\
2x-5y & -4y+3z & \text{signo positivo. En este caso simplemente se quita el}\\
 & & \text{signo junto con los agrupadores}\\

\end{array}
$
\end{center}
\subsection{Ejercicios}
$
\begin{array}{l}
4xy+14z-(5xy+3z)+[5xy-8z-(8xy+3z)]=\\
2ac+(5b-9ac-34c)+8ac+(16c-19ab)-5c=\\
\end{array}
$

\section{Multiplicaciones}
En álgebra se expresa la multiplicación poniendo números y literales juntos, sin signos de suma o resta entre ellos, también encerrándolos entre signos de agrupación y colocándolos sin signos de suma o resta en medio. Así pues, las siguientes expresiones indican multiplicaciones en forma válida.
\begin{center}
$$
\begin{array}{|c|c|c|c|c|}
\hline
 & & & & \\
3x & ab & -4xy & 5x(a+b) & (2x-y)4x-2y)\\
 & & & & \\
\hline
\end{array}
$$
\end{center}
Antes de proceder con la operación se deben aprender las siguientes reglas o leyes:
De los signos:
\begin{center}
$$
\begin{array}{|lcr|}
\hline
(+)(+) & = & (+)\\
(-)(+) & = & (-)\\
(+)(-) & = & (-)\\
(+)(+) & = & (+)\\
\hline
\end{array}
$$
\end{center}

De los exponentes:
\begin{center}
$$
\begin{array}{|l|l|l|l|}
\hline
 & & & \\
a^0=1 & a^1 = a & (a^n)(a^m)=a^{n+m} & (a^n)^m = a^{nm}\\
 & & & \\
\hline
\end{array}
$$
\end{center}
\subsection{Multiplicación de monomios}
Para realizar la multiplicación se operan: signos con signos, números con números y mismas letras con mismas letras, de la forma siguiente. Se hace notar que un término que no tiene signo positivo o negativo visiblemente, es positivo por convención.
\begin{center}
$$
\begin{array}{|c|c|c|}
\hline
 & & \\
(3a)(4)=12a & (-5y)(4x)=-20xy & (6x)(-12x^2) = -72x^3\\
 & & \\
\hline
\end{array}
$$
\end{center}

\subsubsection{Ejercicios}
$$
\begin{array}{l}
3x(5)=\\
4ab(4)=\\
3c(2d)=\\
(5a)(4a^4)=\\
\end{array}
$$

\subsection{Multiplicación de monomio por polinomio}
Se multitplica primero el monimio por el primer término del polinomio de la misma forma que en la multiplicación de un monomio por otro monomio, después se multiplican los signos y luego se multiplica el siguiente término por el monomio, y así hasta completar.
\subsubsection{Ejercicios}
$$
\begin{array}{l}
-5(3+2c)=\\
-3c(2a+5)=\\
-4x(3x+2y-4)=\\
4x(-3x+2y-4xy+5)=\\
\end{array}	
$$

\subsection{Multiplicación de polinomios}
\subsubsection{Ejercicios}
$$
\begin{array}{l}
(2a+5)(3+2c)=\\
(2a+5)(3x+2y-4)=\\
(3x+2y-4)(-3x+2y-4xy+5)=\\
\end{array}
$$

\subsection{Productos notables - Cuadrado de un binomio}
Estos se resuelven de una sencilla forma, para ello se siguen estos pasos:
\begin{enumerate}
\item Se eleva al cuadrado el primer término
\item Se suma: el doble producto del primer término por el segundo término
\item Se suma el cuadrado del segundo término
\end{enumerate}

\subsubsection{Ejercicios}
$$
\begin{array}{l|l|l}
(x+v)^2=   & (2a+3b)^2=   & (5a+b)^2=\\
(2x-z)^2=  & (3b - 5c)^2= & (a-3f)^2=\\
(ab-3)^2=  & (10a+2b)^2=  & (2c+4d)^2=\\
(3y+5z)^2= & (-5a+b)^2=   & (xy-3)^2=\\
(-a+2)^2=  & (-3b-4c)^2=  & (x-8)^2=
\end{array}
$$

\subsection{Productos notables - Binomios conjugados}
Se reconocen por tener cantidades iguales y tener UN signo diferente.
Estos se resuelven rápidamente siguiendo estos pasos:
\begin{enumerate}
\item Se calcula el cuadrado del primer término.
\item Se calcula el cuadrado del segundo término.
\item Se coloca signo \emph{negativo} para el término que tiene diferente signo y signo \emph{positivo} para el término que tiene signos idénticos.
\end{enumerate}

\subsubsection{Ejercicios}
$$
\begin{array}{l|l|l}
(x-v)(x+v) = & (2a+3b)(2a+3b)= & (5a+b)(5a-b)=\\
(2x-z)(2x+z)= & (3b - 5c)(3b-5c)= & (a-3f)(a+3f)=\\
(ab-3)(ab+3)= & (10a+2b)(10a-2b)=  & (2c+4d)(2c-4d)=\\
(3y+5z)(3y-5z)= & (-5a+b)(5a+b)= & (xy-3)(xy+3)=\\
(-a+2)(a+2)= & (-3b-4c)^2=  & (x-8)(x+8)=
\end{array}
$$

\section{Divisiones}
En álgebra se expresa la división de la siguiente manera:

\section{Fracciones}

\section{Ecuaciones enteras de primer grado}
\section{Sistemas de ecuaciones de primer grado}
\section{Ecuaciones fraccionarias de primer grado}
\section{Factorización}
\section{Ecuaciones de segundo grado}
