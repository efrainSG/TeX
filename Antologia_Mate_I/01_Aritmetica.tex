\chapter{Aritmética}

\section{Operaciones con fracciones}

\subsection{¿Qué son las fracciones o quebrados?}
Son números que podemos expresar de la siguiente manera: 
\begin{equation}
\dfrac{numerador}{denominador}
\end{equation}
Donde ambos números son enteros (sin punto decimal) y ambos números pueden ser positivos o negativos, diferentes o iguales. La única condición es que el denominador sea diferente de cero.\\
El denominador indica la cantidad de partes en que algo se divide, y el numerador indica cuántas partes de esas divisiones son tomadas, por ejemplo:
\begin{equation*}
\dfrac{4}{7} \rightarrow \text{Se divide en siete partes y se toman cuatro de éstas.}
\end{equation*}

\subsection{Suma de fracciones}
Si las fracciones a sumar tienen el mismo denominador, entonces únicamente se suman los numeradores y en el resultado se coloca como numerador la suma y como denominador el mismo que tienen las fracciones a sumar.\\
\begin{equation*}
\dfrac{5}{6}+\dfrac{3}{6}+\dfrac{4}{6}=\dfrac{12}{6}
\end{equation*}
Si las fracciones tienen diferente denominador, la forma de resolver se puede organizar en tres partes (la explicación se acompaña de un ejemplo):
\begin{equation*}
\dfrac{5}{6}+\dfrac{3}{8}+\dfrac{7}{3}=
\end{equation*}
\begin{description}
\item[Calcular mínimo común múltiplo]:
\begin{enumerate}
\item Se toman por aparte todos los denominadores separados por comas.\\
\begin{equation*}
6, 8, 3
\end{equation*}
\item Se busca el número entero más pequeño que divida exactamente por lo menos a uno de los denominadores.\\
\begin{equation*}
6, 8, 3 \text{ }\div \text{ } 2
\end{equation*}
\item Se dividen uno a uno los denominadores entre el número elegido y se reemplazan los que puedan ser dividos por el resultado de esas divisiones.\\
\begin{equation*}
6, 8, 3 \text{ }\div \text{ } 2 \Rightarrow 3, 4, 3
\end{equation*}
\item Se repite el proceso hasta que todos los denominadores son cambiados por un $1$.\\
\begin{eqnarray*}
3, 4, 3 \text{ }\div \text{ } 2 \Rightarrow 3, 2, 3\\
3, 2, 3 \text{ }\div \text{ } 2 \Rightarrow 3, 1, 3\\
3, 1, 3 \text{ }\div \text{ } 3 \Rightarrow 1, 1, 1
\end{eqnarray*}
\item Se multiplican todos los números utilizados para dividir los denominadores y este resultado será el \emph{mínimo común múltiplo} que será el nuevo \emph{denominador común}.
\begin{equation*}
2 \times 2 \times 2 \times 3 = 24 \Rightarrow \dfrac{5}{6}+\dfrac{3}{8}+\dfrac{7}{3}=\dfrac{}{24}
\end{equation*}
\end{enumerate}
\item[Calcular numeradores]:
\begin{enumerate}
\item Se divide el denominador común entre el primero de los denominadores originales y se multiplican por su numerador.
\begin{equation*}
24 \div 6 = 4 \rightarrow 4\times 5 = 20 \Rightarrow \dfrac{5}{6}+\dfrac{3}{8}+\dfrac{7}{3}=\dfrac{20+}{24}
\end{equation*}
\item Se repite el paso anterior con cada una de las fracciones originales.
\begin{equation*}
\dfrac{5}{6}+\dfrac{3}{8}+\dfrac{7}{3}=\dfrac{20+9+56}{24}
\end{equation*}
\end{enumerate}
\item[Realizar la suma]: Se suman los nuevos numeradores y se colocan como el numerador resultante mientras que el denominador común se coloca como denominador resultante.
\begin{equation*}
\dfrac{5}{6}+\dfrac{3}{8}+\dfrac{7}{3}=\dfrac{20+9+56}{24}=\dfrac{85}{24}
\end{equation*}
\end{description}

\subsection{Resta de fracciones}
Si las fracciones a restar tienen el mismo denominador, entonces únicamente se restan los numeradores y en el resultado se coloca como numerador la resta y como denominador el mismo que tienen las fracciones a restar.
\begin{equation*}
\dfrac{10}{6}-\dfrac{3}{6}-\dfrac{4}{6}=\dfrac{3}{6}
\end{equation*}
Si las fracciones tienen diferente denominador, la forma de resolver se puede organizar en tres partes (la explicación se acompaña de un ejemplo):
\begin{equation*}
\dfrac{5}{6}-\dfrac{3}{8}-\dfrac{1}{3}=
\end{equation*}
\begin{description}
\item[Calcular mínimo común múltiplo]:
\begin{enumerate}
\item Se toman por aparte todos los denominadores separados por comas.
\begin{equation*}
6, 8, 3
\end{equation*}
\item Se busca el número entero más pequeño que divida exactamente por lo menos a uno de los denominadores.
\begin{equation*}
6, 8, 3 \text{ } \div \text{ }2
\end{equation*}
\item Se dividen uno a uno los denominadores entre el número elegido y se reemplazan los que puedan ser dividos por el resultado de esas divisiones.
\begin{equation*}
6, 8, 3 \text{ }\div\text{ }2 \Rightarrow 3, 4, 3
\end{equation*}
\item Se repite el proceso hasta que todos los denominadores son cambiados por un $1$.
\begin{eqnarray*}
3, 4, 3 \text{ }\div\text{ }2 \Rightarrow 3, 2, 3\\
3, 2, 3 \text{ }\div\text{ }2 \Rightarrow 3, 1, 3\\
3, 1, 3 \text{ }\div\text{ }3 \Rightarrow 1, 1, 1
\end{eqnarray*}
\item Se multiplican todos los números utilizados para dividir los denominadores y este resultado será el \emph{mínimo común múltiplo} que será el nuevo \emph{denominador común}.
\begin{equation*}
2 \times 2 \times 2 \times 3 = 24 \Rightarrow \dfrac{5}{6}-\dfrac{3}{8}-\dfrac{1}{3}=\dfrac{}{24}
\end{equation*}
\end{enumerate}
\item[Calcular numeradores]:
\begin{enumerate}
\item Se divide el denominador común entre el primero de los denominadores originales y se multiplican por su numerador.
\begin{equation*}
24 \div 6 = 4 \rightarrow 4\times 5 = 20 \Rightarrow \dfrac{5}{6}-\dfrac{3}{8}-\dfrac{7}{3}=\dfrac{20-}{24}
\end{equation*}
\item Se repite el paso anterior con cada una de las fracciones originales.
\begin{equation*}
\dfrac{5}{6}-\dfrac{3}{8}-\dfrac{1}{3}=\dfrac{20-9-8}{24}
\end{equation*}
\end{enumerate}
\item[Realizar la resta]: Se restan los nuevos numeradores y se colocan como el numerador resultante mientras que el denominador común se coloca como denominador resultante.
\begin{equation*}
\dfrac{5}{6}-\dfrac{3}{8}-\dfrac{1}{3}=\dfrac{20-9-8}{24}=\dfrac{3}{24}
\end{equation*}
\end{description}

\subsection{Multiplicación de fracciones}
Esta se realiza de una manera muy sencilla: se multiplican todos los numeradores para obtejer el numerador resultante, y se multiplican todos los denominadores para obtener el denominador resultante, por ejemplo:
\begin{equation*}
\dfrac{3}{5}\times\dfrac{2}{7}\times\dfrac{1}{2}=\dfrac{3\times 2\times 1}{5\times 7\times 2}=\dfrac{6}{70}
\end{equation*}

\subsection{División de fracciones}
Esta operación se realiza multiplicando de forma alternada numeradores y denominadores, tal como en los siguientes ejemplos:
\begin{eqnarray*}
\dfrac{1}{2}\div\dfrac{3}{4}=\dfrac{1\times 4}{2\times 3}=\dfrac{4}{6}\\
\\
\dfrac{1}{4}\div\dfrac{2}{5}\div\dfrac{3}{6}=\dfrac{1\times 5\times 3}{4\times 2\times 6}=\dfrac{15}{36}
\end{eqnarray*}
\subsection{Fracciones equivalentes}
Son grupos de fracciones que tienen el mismo valor aunque sus numeradores y denominadores sean diferentes, y se pueden encontrar multiplicando o dividiendo por un mismo número tanto su numerador como su denominador, por ejemplo:
\begin{eqnarray*}
\dfrac{1}{2}\text{ y }\dfrac{2}{4}\text{ son equivalentes porque }\dfrac{1 \times 2}{2 \times 2} = \dfrac{2}{4}\\
\\
\dfrac{6}{9}\text{ y }\dfrac{2}{3}\text{ son equivalentes porque }\dfrac{6 \div 3}{9 \div 3} = \dfrac{2}{3}
\end{eqnarray*}
\section{Operaciones con razones y proporciones}

\subsection{¿Qué es una razón y qué es una proporción?}
Una razón es la división entre dos números, tal como ocurre con una fracción, mientras que una proporción es la comparación entre dos razones o fracciones donde las dos razones son equivalentes, por ejemplo:
\begin{equation*}
\dfrac{4}{5}:\dfrac{12}{15}
\end{equation*}
Para asegurar que la proporción se cumple, se multiplican de forma cruzada los números, y si ambos resultan iguales, entonces la proporción se cumple.
\begin{equation*}
\dfrac{4}{5}:\dfrac{12}{15}\Rightarrow 4\times 15 : 5\times{12}\Rightarrow 60:60
\end{equation*}
\subsection{Regla de tres directa}
Aplica para proporciones de relación directa, es decir, donde las cantidades aumentan en la misma proporción, por ejemplo:\\
Si se compra una botella de 700 ml a 14 pesos, ¿en cuánto se comprarán 900 ml?\\
Para resolverlo se plantea la proporción de la siguiente manera y se resuelve como sigue:
\begin{equation*}
\dfrac{700}{900}=\dfrac{14}{x}\Rightarrow x=\dfrac{900\times 14}{700} = 18
\end{equation*}
\subsection{Regla de tres inversa}

\section{Operaciones con porcentajes}

\subsection{¿Qués es un porcentaje?}

\section{Miscelanea de problemas}
