Aguas con tu espalda...
Yo me dedico al desarrollo de sistemas ("¡qué novedad! y a mí qué?" dirán muchos :p) y como tal, me paso mucho tiempo sentado ("igual que yo" otros dirán :p) y lamentablemente he padecido dolores de espalda y piernas ("yo también y no me quejo") en forma frecuente a lo largo de mi vida laboral -que no es mucha-

Pues bien... esto me llevó al quirófano, una temidísmia cirugia para remover las dos hernias lumbares que ya tenía y que me impedían caminar en el último mes a mes y medio... así es... entre 30 y 45 días acostado ("qué bien!! yo quiero unas hernias")... sobre mi lado izquierdo, acurrucado porque el dolor no me dejaba estirarme, ni acostrme boca arriba y menor girarme... ir al baño, ir a comer... todas unas proezas...

Gracias a Dios, y al excelente cirujano de nuestro gran IMSS, salí de ahi sin hernias, sin dolor y aquí sigo dando lata.

Pues bien, esto me lleva a escribir en este post unas pequeñas notas que encontré por la red y que tanto ifnromáticos como todos los que pasemos más de 2 horas sentados debemos considerar... y por caro que sea el mobiliario... compárenlo con $150,000 pesos que cuesta la cirugía (no todos tienen Seguridad Social) y 2 a 3 semanas de rehabilitación junto con los medicamentos, el riesgo de desarrollar una fibrosis (una especie de cayo) que les llevaría a una nueva cirugía y la posibilidad de que en la chamba les digan "no puedes trabajar, no me sirves. Bye con tu vida"

8 Características básicas de una buena silla ergonómica:

Verdadero soporte lumbar: muchas sillas “económicas” dicen que tiene soporte lumbar, pero el verdadero soporte lumbar es algo más que tener un acolchado, es el que mantiene la curvatura natural del hueco que se produce en la espalda.
Suave curvatura en cascada: el borde de la silla alivia la presión sobre los vasos sanguíneos de los muslos y previene el entumecimiento de las piernas, los pies fríos y las venas varicosas. El borde delantero del asiento debe inclinarse suavemente hacia abajo y no debe presionar su muslo.
Acolchado: Menos es más. La gente se equivoca al comprar sillas muy acolchadas — con el tiempo, el acolchado se adaptará a la mala postura de su espalda.
Mobilidad: la silla debe poder deslizarse sin esfuerzo para permitirle al cuerpo hacer movimientos de relax
Apoya brazos: Deje que la silla y no la parte superior de su espalda soporte el peso de los brazos mientras trabaja.
Profundidad: Una silla muy profunda puede resultar problemática para una persona de contextura pequeña. Cuando está sentada con la espalda bien apoyada, debería haber suficiente espacio entre el borde de la silla y su rodilla para que quepa un puño cerrado.
Altura: Asegúrese que la silla es suficientemente alta para que sus muslos formen un ángulo de 90° con el piso.
Apoya pies: Considere tener un apoya pies si la silla es demasiado alta. Tener los pies apoyados le ayudará a restablecen la curva natural de su espalda.
Monitores
Características:

Sus colores han de ser claros y mates. Así se evitan reflejos.
Los caracteres tienen que estar bien definidos, con un buen nivel de contraste con respecto al fondo, de tamaño suficiente y con un espacio adecuado entre los renglones.
Esto facilita la legibilidad. Es preferible trabajar con estas características y modificarlas, si se desea, en el momento de la impresión.
La imagen de la pantalla ha de ser estable, sin destellos, reflejos, centelleos o reverberaciones. Un estudio de la Universidad de Santiago ha puesto de manifiesto que el nivel de luminancia de los monitores de rayos catódicos es inestable durante los primeros 20 minutos tras el encendido. Parece aconsejable un precalentamiento de la pantalla para evitar una posible fatiga visual producida por estas variaciones.
Orientable a voluntad. Con el fin de acomodarlo a las posturas que se adopten y para optimizar los ángulos de visión, así como para evitar reflejos.
Regulable en cuanto a brillo y contraste. Para adaptarlos a las condiciones del entorno. Además, los mandos, interruptores y botones deben ser fácilmente accesibles, con el fin de que permitan una sencilla manipulación.
Consejos para usarlo:

Trabaje con monitores que lleven un tratamiento antirreflejo o incorporen un filtro especial. El cristal de los monitores refleja la luz que le llega. Estos destellos son molestos para el ojo, porque reducen la legibilidad y obligan a una constante acomodación de la visión. Hay que tener un especial cuidado en que el filtro no oscurezca demasiado el monitor.
Procure que la pantalla esté siempre limpia. Las huellas y demás suciedades también provocan reflejos. La radiación que emiten algunas pantallas es mínima y no supone ningún peligro. Sin embargo, los campos electroestáticos atraen el polvo, lo que puede afectar a las vías respiratorias e irritar los ojos. Esto puede evitarse con un grado adecuado de humedad en el ambiente, o con un filtro provisto de un cable de conexión a masa.
Si sufre algún problema en la visión, es mejor utilizar una gafa especialmente destinada al uso del ordenador. Consulte al oftalmólogo. Las gafas de sol reducen la capacidad de lectura.
Trabaje con texto negro sobre fondo blanco. Se debe procurar no abusar de los colores.
Sitúe la pantalla a una distancia entre 50 y 60 centímetros. Nunca a menos de 40 centímetros.
La parte superior de la pantalla debe estar a una altura similar a la de los ojos, o ligeramente más baja. El monitor se sitúa así en la zona óptima de visión, comprendida entre los cinco y los 35 grados por debajo de la horizontal visual, y desde la cual se contempla todo sin ningún esfuerzo. De esta forma, la vista no se resiente y se evitan posturas lesivas.
También es conveniente usar un atril para los documentos. Colocándolo a una distancia equivalente a la pantalla y a su misma altura. De esta forma no se baja y se sube constantemente la cabeza para mirar y se reduce la fatiga visual.
Ubicación:

La pantalla ha de colocarse perpendicular a las ventanas. Nunca enfrente o de espaldas a ellas. En el primer caso, al levantar la vista, se pueden producir deslumbramientos. En el segundo, los reflejos de la luz natural sobre el cristal son inevitables.
