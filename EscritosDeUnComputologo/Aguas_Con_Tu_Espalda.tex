\chapter{Aguas con tu espalda}
\section{Prólogo}
No es algo nuevo, se ha ido dando en el curso de los años comenzando con trabajos en oficina con largas jornadas que luego se redujeron a ocho horas diarias y ahora nuevamente se dan en jornadas de diez y en ocasiones de 12 horas por día, a veces incluidos fines de semana.

El avance tecnológico que debería ayudarnos a tener vidas más cómodas nos ha conducido a una vida sedentaria cada vez más prolongada con las consencuencias para la salud personal. Entre este tipo de trabajos está el de los ingenieros de software: programadores, arquitectos, desarrolladores, ingenieros de pruebas, todos quienes trabajan en mayor o menor medida con equipos de cómputo.

Yo me dedico al desarrollo de sistemas y como tal, me paso mucho tiempo sentado, y lamentablemente he padecido dolores de espalda y piernas en forma frecuente a lo largo de mi vida laboral.

Pues bien, esto me llevó al quirófano, una temida cirugia para remover las dos hernias lumbares que ya tenía y que me impedían caminar en el último mes a mes y medio, así es, entre 30 y 45 días acostado únicament sobre mi lado izquierdo, acurrucado porque el dolor no me dejaba estirarme, acostrme boca arriba ni girarme. Toda actividad que me requiriera estar en pie o sentado era algo titánico.

Gracias a Dios, y al excelente cirujano de nuestro gran IMSS, salí de ahi sin hernias, sin dolor y puedo hacer casi todas mis actividades sin dolor.

Esto me lleva a escribir aquí unas pequeñas notas que encontré por la red y que tanto infromáticos como todos los que pasemos más de 2 horas sentados debemos considerar. Y por caro que sea el mobiliario hay que compárarlo con los \$150,000 pesos que cuesta la cirugía (no todos tienen Seguridad Social) y 2 a 3 semanas de rehabilitación junto con los medicamentos, el riesgo de desarrollar una fibrosis que les llevaría a una nueva cirugía y las posibles consecuencias negativas en varios aspectos de la vida, comenzando por el laboral.

\section{Una buena silla ergonómica}
\subsection{Características}
\begin{itemize}
\item Verdadero soporte lumbar: muchas sillas “económicas” dicen que tiene soporte lumbar, pero el verdadero soporte lumbar es algo más que tener un acolchado, es el que mantiene la curvatura natural del hueco que se produce en la espalda.
\item Suave curvatura en cascada: el borde de la silla alivia la presión sobre los vasos sanguíneos de los muslos y previene el entumecimiento de las piernas, los pies fríos y las venas varicosas. El borde delantero del asiento debe inclinarse suavemente hacia abajo y no debe presionar su muslo.
\item Acolchado: Menos es más. La gente se equivoca al comprar sillas muy acolchadas — con el tiempo, el acolchado se adaptará a la mala postura de su espalda.
\item Mobilidad: la silla debe poder deslizarse sin esfuerzo para permitirle al cuerpo hacer movimientos de relax.
\item Apoya brazos: Deje que la silla y no la parte superior de su espalda soporte el peso de los brazos mientras trabaja.
\item Profundidad: Una silla muy profunda puede resultar problemática para una persona de contextura pequeña. Cuando está sentada con la espalda bien apoyada, debería haber suficiente espacio entre el borde de la silla y su rodilla para que quepa un puño cerrado.
\item Altura: Asegúrese que la silla es suficientemente alta para que sus muslos formen un ángulo de 90° con el piso.
\item Apoya pies: Considere tener un apoya pies si la silla es demasiado alta. Tener los pies apoyados le ayudará a restablecen la curva natural de su espalda.
\end{itemize}

\section{Monitores y pantallas}
\subsection{Características}
\begin{itemize}
\item Sus colores han de ser claros y mates. Así se evitan reflejos.
\item Los caracteres tienen que estar bien definidos, con un buen nivel de contraste con respecto al fondo, de tamaño suficiente y con un espacio adecuado entre los renglones. Esto facilita la legibilidad. Es preferible trabajar con estas características y modificarlas, si se desea, en el momento de la impresión.
\item La imagen de la pantalla ha de ser estable, sin destellos, reflejos, centelleos o reverberaciones. Un estudio de la Universidad de Santiago ha puesto de manifiesto que el nivel de luminancia de los monitores de rayos catódicos es inestable durante los primeros 20 minutos tras el encendido. Parece aconsejable un precalentamiento de la pantalla para evitar una posible fatiga visual producida por estas variaciones.
\item Orientable a voluntad. Con el fin de acomodarlo a las posturas que se adopten y para optimizar los ángulos de visión, así como para evitar reflejos.
\item Regulable en cuanto a brillo y contraste. Para adaptarlos a las condiciones del entorno. Además, los mandos, interruptores y botones deben ser fácilmente accesibles, con el fin de que permitan una sencilla manipulación.
\end{itemize}

\subsection{Consejos para usarlo}
Trabaje con monitores que lleven un tratamiento antirreflejo o incorporen un filtro especial. El cristal de los monitores refleja la luz que le llega. Estos destellos son molestos para el ojo, porque reducen la legibilidad y obligan a una constante acomodación de la visión. Hay que tener un especial cuidado en que el filtro no oscurezca demasiado el monitor.

Procure que la pantalla esté siempre limpia. Las huellas y demás suciedades también provocan reflejos. La radiación que emiten algunas pantallas es mínima y no supone ningún peligro. Sin embargo, los campos electroestáticos atraen el polvo, lo que puede afectar a las vías respiratorias e irritar los ojos. Esto puede evitarse con un grado adecuado de humedad en el ambiente, o con un filtro provisto de un cable de conexión a masa.

Si sufre algún problema en la visión, es mejor utilizar una gafa especialmente destinada al uso del ordenador. Consulte al oftalmólogo. Las gafas de sol reducen la capacidad de lectura.

Trabaje con texto negro sobre fondo blanco. Se debe procurar no abusar de los colores.

Sitúe la pantalla a una distancia entre 50 y 60 centímetros. Nunca a menos de 40 centímetros.

La parte superior de la pantalla debe estar a una altura similar a la de los ojos, o ligeramente más baja. El monitor se sitúa así en la zona óptima de visión, comprendida entre los cinco y los 35 grados por debajo de la horizontal visual, y desde la cual se contempla todo sin ningún esfuerzo. De esta forma, la vista no se resiente y se evitan posturas lesivas.
También es conveniente usar un atril para los documentos. Colocándolo a una distancia equivalente a la pantalla y a su misma altura. De esta forma no se baja y se sube constantemente la cabeza para mirar y se reduce la fatiga visual.

\subsection{Ubicación}
La pantalla ha de colocarse perpendicular a las ventanas. Nunca enfrente o de espaldas a ellas. En el primer caso, al levantar la vista, se pueden producir deslumbramientos. En el segundo, los reflejos de la luz natural sobre el cristal son inevitables.
