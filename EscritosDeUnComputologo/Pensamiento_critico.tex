¿Pensamiento crítico?
Algo donde siento que se puede ejemplificar la búsqueda del desarrollo individual y la construcción de un pensamiento crítico como herramienta fundamental es con las posturas que cada quien toma respecto a ciertos temas, como por ejemplo decir que sería mejor un mundo sin religión, algunos dirán que es cierto, otros diremos que no. Si el que dice que no es cierto porque las religiones buscan la tolerancia y el respeto y en su proceder no actúa según lo que pregona, argumenta y defiende ¿es coherente, íntegro y humilde? O el que argumenta que las religiones son intolerantes, vacías, impositivas y manipuladoras y actúa tratando de imponer esa idea por medio de mensajes breves sin argumentos válidos o tan pronto siente el intento de una defensa contra su forma de pensar, salta al instante ¿a caso demuestra coherencia y disposición a encontrar razones?. Santiago Apóstol (santiago 2:18) escribió "Y sería fácil decirle a uno: 'Tú tienes fe, pero yo tengo obras. Muéstrame tu fe sin obras, y yo te mostraré mi fe a través de las obras.'" que para el contexto del tema que estamos tratando, expresa la importancia de la integridad, no solo intelectual, sino personal (una persona íntegra lo será en pensamientos-palabras-obras, por tanto, si uno es íntegro, lo será también intelectualmente hablando).

Pienso (ya me dirán si estoy errado) que en estos puntos de la vida cotidiana es donde:
Podemos encontrar en nosotros mismos, ya sea viéndonos directamente o viéndonos como reflejo en los demás, el grado de humildad (para reconocer nuestros errores), integridad (la concordancia pienso-hago-digo), la valentía (para decir "no coincido contigo por estoy esto otro más")
Viendo tales niveles, darnos a la investigación crítica (saber dónde buscar, qué buscar y como mencionaban, confrontar las ideas): "Yo solo sé que no sé nada, entonces ya sé algo, sé que nada sé"
Analizando, buscando, confrontando, reconociendo nuestros límites, argumentando, teniendo valor: así iremos formando un pensamiento crítico que nos ayude a crecer, a ser tolerantes, respetuosos, sin ser dejados.