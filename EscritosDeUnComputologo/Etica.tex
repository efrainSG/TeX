Ética ¿Pasada de moda o moda que no debe pasar?
Plagio: ¿astucia, comodidad o deshonestidad?
¿Qué es el plagio? ¿Qué efectos puede tener? ¿Cómo me afecta si lo realizo? ¿Puedo hacer algo para promoverlo? ¿Puedo hacer algo en contra? ¿Me afecta si lo sufro? Estas son algunas de las preguntas que podemos plantear, ver y analizar el plagio desde el punto de vista de quien lo realiza como de quien lo sufre, analizar brevemente sus consecuencias y su impacto en la vida diaria.
¿Qué es el plagio? ¿Qué efectos puede tener? ¿Cómo me afecta si lo realizo? ¿Puedo hacer algo para promoverlo? ¿Puedo hacer algo en contra? ¿Me afecta si lo sufro? Estas son algunas de las preguntas que podemos plantear, ver y analizar el plagio desde el punto de vista de quien lo realiza como de quien lo sufre, analizar brevemente sus consecuencias y su impacto en la vida diaria.
Si es bueno o es malo, lo iremos dilucidando a lo largo del trabajo mediante situaciones vividas por Raúl Rojas Soriano y expresadas en su trabajo formación de Investigadores educativos de 1992, junto a situaciones de las que he sabido.
La RAE en su diccionario lo define como Acción y efecto de plagiar (copiar obras ajenas) y como americanismo que refiere a secuestrar a alguien. Ya de entrada no suena muy bien eso de "secuestrar". Por el lado de "copiar obras ajenas", más que copiarlas, refiere a transcribir o parafrasear la obra de otra persona o parte de ella sin decir que esa obra o esa parte pertenecen al trabajo de otro.
Raúl Rojas Soriano, en su trabajo Formación de investigadores educativos de 1992, platica la situación donde él fungía como parte de un jurado y que una concursante presentó un trabajo como propio, más tarde, Rojas se encontró con una conocida que le obsequió un ejemplar de uno de sus trabajos y al leerlo se dio cuenta que el trabajo de la concursante era un plagio, por lo que antes de deliberar y para evitar una situación vergonzosa para la concursante, le pidieron en privado que se retirara del concurso. Otra situación de este tipo la supe por unos compañeros de universidad que supieron que una maestra solicitó una serie de proyectos de fin de semestre, los cuales terminó presentando como propios para una evaluación que tuvo a su vez ella, resulta que uno de sus estudiantes descubrió tal situación e interpuso una demanda, porque él registró su proyecto a fin de tener derechos de autoría intelectual.
Como nos damos cuenta, el plagio no es bueno, y de afectarnos en caso de incurrir en él, seguro que sí nos afectará, tal vez no se descubra de inmediato, pero en cuanto suceda habrá consecuencias.
Esto nos habla sobre la importancia de la honestidad, de la honradez, sobre la empatía que debe existir entre nosotros, porque si yo hago algo que me cuesta, como decimos, tiempo, dinero y esfuerzo, con toda sinceridad y justicia no voy a querer que otro me lo robe (en el folklore mexicano: "agandalle", "madrugue", "fusile", "haga reverencia con sombrero ajeno"). Y si no lo queremos nos lo hagan ¿por qué hacerlo a otros? La situación cambia cuando voluntariamente lo ofrecemos a la comunidad para su uso.
Ética, Comprensión y Humanidad
Hablando de empatía, honradez, honestidad, es interesante lo que plantea Edgar Morín en su trabajo Los siete saberes necesarios para la educación del futuro en su capítulo VI, pues siendo la empatía la clave para comprender al otro, Edgar expone que vivimos una paradoja, por un lado tomamos conciencia de solidaridad, sabemos de personas en otro lado del planeta, los estudiamos y somos estudiados, comprendemos sus culturas, mientras que por el otro lado, entre nosotros y los que nos rodean existe cada vez más incomprensión, existe un "ruido" cada vez mayor que provoca malos entendidos, ambigüedad en lo que decimos, de forma tal que lo que decimos en un sentido los demás suelen darle otra interpretación. ¿Cuántas veces no se ha escuchado decir a alguien (o uno mismo) "yo no quise decir eso" o "lo que quise decir" o "no me mal interpretes" o "no me entiendes"? Recuerdo que un escritor dijo "cada quien llama barbarie a lo que les extraño", esto nos muestra que solemos no comprender a personas de otras culturas, ideas o filosofías, es difícil comprender a otros cuando uno mismo no se comprende (Morín, E, los siente saberes necesarios para la educación del futuro, Cap. VI) "Temet nosce" ("conócete a ti mismo") reza un cartel colocado en el dintel de una puerta de cierto personaje y es lo más difícil de realizar pero es la clave para ir comprendiendo a los demás, precisamente es lo opuesto a conocerse uno mismo lo que lleva a, como dice Morín nuevamente, la indiferencia, el ego, etno y sociocentrismo (creerse a uno mismo, a su raza y su sociedad como el centro de todo), ampliar el abandono de la disciplina y obligación, la autoglorificación, autojustificación y a adjudicar a los demás los males que nos aquejan ("tú tienes la culpa", "no me entiendes", "no me das nada"); el trabajo de conocerse a uno mismo es largo y profundo, es difícil, intenso, agotador, es ampliar la visión y lo contrario, es sencillo: reducir todo lo complejo a su característica más notoria según el momento (dice el refrán "el árbol no deja ver el bosque"), sin embargo, conocerse un poquito cada vez más es una tarea gratificante en muchos aspectos, los réditos son mayores.} Bien, si nos conocemos un poquito más veremos que tenemos cualidades y virtudes, defectos, manías y problemas y que no somos "moneditas de oro" entonces si a nosotros mismos nos cuesta dominar esos puntos álgidos que tenemos ¿cómo es que exigimos que otros hagan rápidamente lo que a nosotros nos cuesta muchísimo trabajo? Hay que dar hasta que duela, decía la madre Teresa de Calcuta, el Señor Jesús enseñaba "Da al que te pide, y al que te quita lo tuyo, no se lo reclames. Traten a los demás como quieren que ellos les traten a ustedes. No juzguen y no serán juzgados; no condenen y no serán condenados; perdonen y serán perdonados. Den, y se les dará; se les echará en su delantal una medida colmada, apretada y rebosante. Porque con la medida que ustedes midan, serán medidos ustedes. ¿Y por qué te fijas en la pelusa que tiene tu hermano en un ojo, si no eres consciente de la viga que tienes en el tuyo? ¿Cómo puedes decir a tu hermano: 'Hermano, deja que te saque la pelusa que tienes en el ojo', si tú no ves la viga en el tuyo? Hipócrita, saca primero la viga de tu propio ojo para que veas con claridad, y entonces sacarás la pelusa del ojo de tu hermano." (Lc. 6: 30-31, 37-38, 41-42).
La comprensión, su ética es exactamente de esta forma, pide comprender de forma desinteresada, sin esperar ser comprendido, comprender la incomprensión, argumentar y refutar en vez de excomulgar y anatematizar, la comprensión ni excusa ni acusa. Sabiendo comprender estaremos en vías de humanizar las relaciones humanas. Además, la comprensión hacia los demás necesita la conciencia de la complejidad humana (Morín, 1999) porque como humanos no somos entes simples, sino llenos de facetas, pareceres, humores y cambiamos con cada situación y decisión que tomamos.
Morín también explica que “La verdadera tolerancia no es indiferente a las ideas o escepticismos generalizados”, involucra convicción, fe, elección ética, aceptación y nos recuerda cuatro grados de tolerancia: respetar el derecho a proferir un propósito que nos parece innoble, nutrirse de opiniones diversas y antagónicas, lo opuesto a una idea profunda es otra idea profunda y la conciencia de las enajenaciones humanas. La tolerancia vale para las ideas, no para los insultos, agresiones o actos homicidas.
Inevitablemente, todo esto me suena conocido y estoy seguro que no soy el único con esa sensación, de hecho se puede encontrar algo análogo escrito hace siglos ya, por san Pablo "El amor es paciente, es servicial; el amor no es envidioso, no hace alarde, no se envanece, no procede con bajeza, no busca su propio interés, no se irrita, no tienen en cuenta el mal recibido, no se alegra de la injusticia, sino que se regocija con la verdad. El amor todo lo disculpa, todo lo cree, todo lo espera, todo lo soporta." ¿Por qué me suena familiar? Si pensamos en lo que es la ética, lo que es el respeto, la tolerancia de lo que se ha hablado, la honestidad, empatía que Morín y Rojas mencionan en sus trabajos, veremos que en síntesis es equivalente a decir "trata a los demás como quieras ser tratado", "ama a tu prójimo como a ti mismo", sabemos que cuando verdaderamente se ama, nunca intentaremos lastimar, engañar, robar a quien amamos.
Democracia ¿Qué tienes que ver con nosotros?
Tolerar a los demás habla de que todos tenemos defectos, de que cada uno es diferente a los demás… diversidad de personas, de razas, credos, pensares, de ahí que la real forma de ponernos de acuerdo, de soportarnos y no sentirnos agraviados radica en el conceso, en encontrar lo que la mayoría puede requerir sin excluir a lo que la minoría pide, siempre en el marco del crecimiento y respeto de las comunidades ¿no es esto la democracia?
Morín nos explica que como tal, y nacida de sociedades complejas formadas por individuos complejos, resulta en un sistema complejo que tiene su fuerza y debilidad en los antagónicos, es la unión entre lo unido y lo desunido, tolera y se alimenta de conflictos, es plural y debe conservarse así para seguir existiendo.
Conclusiones
Pues analizando los textos anteriores, sus conceptos, similitudes, valores que plantean podemos idear formas para desarrollar conciencias de respeto, conciencias que lleven al crecimiento de una ética que se base en la riqueza de la diversidad, como por ejemplo:
Al inicio de los ciclos escolares, entre profesores y alumnos es adecuado establecer un acuerdo en las reglas que operarán a lo largo del curso, de esta forma se fomenta el respeto a las decisiones tomadas en grupo.
Dejar de lado la costumbre de mentir para “salir del apuro”, incluso dejar de recurrir a las “mentiras piadosas”, ya que esto es negativo para fomentar la honestidad y honradez.
Proporcionar los materiales de fuentes legales, si son películas, aunque sea alquiladas, si son libros, buscar ejemplares, en bibliotecas.
Fomentar el trabajo en equipo y entre los equipos que existan desacuerdos hay que intervenir, no para decir “él tiene la razón”, sino para generar empatía entre los involucrados, de forma que se abran al diálogo y lleguen a acuerdos que los beneficien.
Los beneficios que se pueden alcanzar serán mayores, ya que tanto profesores como alumnos iremos educándonos y re-educándonos en la verdad, la honestidad, el aprecio por las diferentes opiniones, el diálogo, la empatía y esto se irá poniendo en práctica en los ambientes que nos rodean.

Código de ética
Toda persona que desee realizar una investigación educativa con el estricto rigor científico deberá cumplir al pie de la letra por lo menos los siguientes preceptos:

Ser una persona honesta en todo su contexto, desde el entorno personal hasta el entorno laboral y académico.
Ser coherente entre lo que se piensa, se dice, se hace y se es.
Mantener una actitud de pensamiento crítico fuerte, en especial ser humilde para reconocer aquellos puntos en los que hay que corregirse y valeroso para defenderse y defender a otros cuando la ocasión lo requiera.
Ser respetuoso con el manejo de las fuentes de consulta y citar conforme a la norma APA y en general dar crédito correspondiente a quien sea autor de las ideas.
Expresar de manera concreta cuáles son los objetivos de cualquier trabajo que se realice, ya sea de investigación, como de otra índole.
No cambiar los objetivos de la investigación a medida que se avanza en ella, ya que es muestra de inconsistencia con uno mismo y da pauta a que la confianza otorgada demerite.
Determinar el alcance del trabajo, sin importar el tipo de proyecto que se realice (esto incluye a los trabajos de investigación y de desarrollo) desde un inicio.
Respetar los datos obtenidos a lo largo de la investigación y no manipular la información de acuerdo a la propia conveniencia, esto es, no sacar de contexto las ideas ni resumirlas o recortarlas de manera que su significado y marco de contexto sea susceptible a interpretaciones ambiguas.
Respetar la confidencialidad del proyecto que se esté realizando.
Tomar en cuenta las diferentes posturas y enfoques que hay sobre el objeto de estudio.

Referencias
Rojas, R. (1992) Formación de investigadores educativos. México: Editorial Plaza y Valdés.
Morín, E. (1999) Los siete saberes necesarios para la educación del futuro. Correo de la UNESCO
Sagrada Biblia, Santo Evangelio según san Lucas
Sagrada Biblia, Primera carta de san Pablo a los Corintios