\chapter{Valores y Democracia: ¿una dupla adecuada?}
\section{Introducción}
Piensa en algo o alguien que sea muy importante para ti. Piensa cómo sería tu vida sin ello. ¿De qué serías capaz por conseguir eso que te es muy importante? ¿Puedes ponerle un precio?, entonces, ¿Es preciado o es valioso? Si a lo que pensaste le puedes poner un precio, sería algo preciado, si no pudiste ponerle precio, es algo valioso. ¿Qué será mejor, algo preciado o algo valioso? Pues esas “sutiles diferencias serán abarcadas en la primera parte de este trabajo, junto con otros conceptos interesantes.

Ahora algo un poco delicado, ¿Te consideras alguien que tiene creencias espirituales o no?, si la respuesta no es ni “si” ni “no”, define qué si y qué no entran en tus creencias. Así como puedes decir “si”, “no”, “más o menos” o “a veces”, también hay conceptos que nos pueden ayudar a saber dónde encajamos. En la segunda parte aclararemos brevemente esos conceptos, pues el objetivo de este trabajo no es excluir a nadie, sino incluir y ayudar a clarificar algunas cosas que causan confusión.

Algo más, ¿Cómo crees que debería ser la educación? ¿Se cumple lo que constitucionalmente se dicta para la educación? ¿Conoces lo que marca la constitución? Pues respecto de eso trataré en la tercera parte, para finalmente descubrir si hay alguna forma de mejorar, si involucrar los valores con la educación puede ayudar o no, si es constitucional o se violan las leyes.
\section{Diferencia entre valor, moral, principio, regla y costumbre.}
Retomando de las primeras preguntas de la introducción algunos términos y para mantener cierta imparcialidad, diré que algo valioso será aquello que tenga o represente un valor para nosotros y que de acuerdo a la RAE (2010), “valor (Del lat. valor, -ōris). Grado de utilidad o aptitud de las cosas, para satisfacer las necesidades o proporcionar bienestar o deleite. Cualidad que poseen algunas realidades, consideradas bienes, por lo cual son estimables. Los valores tienen polaridad en cuanto son positivos o negativos, y jerarquía en cuanto son superiores o inferiores.”, mientras que algo preciado es algo que tiene un precio, que está definido como “1. adj. Precioso, excelente y de mucha estimación”. Otro término que se le parece o que tiene relación sería principio y si hemos oído el término de principio físico o principio matemático o principio social, salen otros más, el de ley, regla y costumbre, mismos que la RAE (2010) define como “principio. (Del lat. principĭum). Base, origen, razón fundamental sobre la cual se procede discurriendo en cualquier materia. Cada una de las primeras proposiciones o verdades fundamentales por donde se empiezan a estudiar las ciencias o las artes. Norma o idea fundamental que rige el pensamiento o la conducta.”, “ley. (Del lat. lex, legis). Regla y norma constante e invariable de las cosas, nacida de la causa primera o de las cualidades y condiciones de las mismas. Precepto dictado por la autoridad competente, en que se manda o prohíbe algo en consonancia con la justicia y para el bien de los gobernados. En el régimen constitucional, disposición votada por las Cortes y sancionada por el jefe del Estado. Cada una de las disposiciones comprendidas, como última división, en los títulos y libros de los códigos antiguos, equivalentes a los artículos de los actuales.”, “regla. (Del lat. regŭla). Aquello que ha de cumplirse por estar así convenido por una colectividad. Conjunto de preceptos fundamentales que debe observar una orden religiosa. Estatuto, constitución o modo de ejecutar algo. En las ciencias o artes, precepto, principio o máxima. Razón que debe servir de medida y a que se han de ajustar las acciones para que resulten rectas. Orden y concierto invariable que guardan las cosas naturales.”, “costumbre. (Del lat. *cosuetumen, por consuetūdo, -ĭnis). Hábito, modo habitual de obrar o proceder establecido por tradición o por la repetición de los mismos actos y que puede llegar a adquirir fuerza de precepto. Aquello que por carácter o propensión se hace más comúnmente. Conjunto de cualidades o inclinaciones y usos que forman el carácter distintivo de una nación o persona.”. Algo que es frecuente, es relacionar “valor” con “moral” y ésta con “religión”, pero veamos qué dice la RAE (2010) sobre “moral” y “religión”: “moral. (Del lat. morālis). Perteneciente o relativo a las acciones o caracteres de las personas, desde el punto de vista de la bondad o malicia. Que no pertenece al campo de los sentidos, por ser de la apreciación del entendimiento o de la conciencia. Prueba, certidumbre moral. Que no concierne al orden jurídico, sino al fuero interno o al respeto humano. Aunque el pago no era exigible, tenía obligación moral de hacerlo. Ciencia que trata del bien en general, y de las acciones humanas en orden a su bondad o malicia. Conjunto de facultades del espíritu, por contraposición a físico. Ánimos, arrestos. Estado de ánimo, individual o colectivo. En relación a las tropas, o en el deporte, espíritu, o confianza en la victoria.”, “religión. (Del lat. religĭo, -ōnis). Conjunto de creencias o dogmas acerca de la divinidad, de sentimientos de veneración y temor hacia ella, de normas morales para la conducta individual y social y de prácticas rituales, principalmente la oración y el sacrificio para darle culto. Virtud que mueve a dar a Dios el culto debido. Profesión y observancia de la doctrina religiosa. Obligación de conciencia, cumplimiento de un deber. La religión del juramento. Orden (‖ instituto religioso)”.

En resumen, podemos decir que un valor es algo que es deseable, que es bueno o que proporciona bienestar, un valor puede tener un grado de estimación, desde completamente no estimado hasta completamente estimado, además son distintos de los principios, leyes y reglas en cuanto que un principio forma una base de la cual parten diversas cosas y un valor no, un valor es poseído por algo y es buscado, es un fin, y en cuanto a su aplicación es un medio también, las leyes son reglas y normas entre grupos a las cuales es deseable cumplir para el bienestar del grupo y que suele ser establecida por representantes de dicha comunidad o por la comunidad completa, mientras que las reglas, definen lo que se debe cumplir o el modo en que se realizan diversas actividades al interior de una colectividad. Y aunque una costumbre, que es un hábito, puede llegar a convertirse en un precepto y forman parte de la identidad de una persona o grupo de personas, no es obligado que se transforme en ley o regla. Dijimos que “valor” suele relacionarse con “religión” y con “moral”, así que, de acuerdo a las definiciones, la moral se entiende como la valoración que se hace de las acciones y que es fundamentalmente de manera interna, y aunque pueden ser coincidentes por el general de las personas, no es en la misma medida para todos y son coadyuvantes para la adquisición de los valores; por parte de la religión, son conjunto de creencias, principios y reglas que debe cumplir un grupo en relación a una divinidad con la finalidad de darle culto, y en ese sentido de reglas y principios, encontramos que existen cosas deseables a fin de trascender de acuerdo a esas creencias. Por tanto, podemos decir que aunque la religión define reglas y principios con metas alcanzables, y la moral dicta a cada persona la forma de comportarse correctamente a su particular forma de ver y ambos definen elementos deseables, no es de su exclusividad el establecer lo que son los valores, sino cuáles de estos son más o menos importantes para el bienestar particular y colectivo.

Bien, eso suena bien, pero… ¿Qué es lo que consideran los entendidos del pensamiento como valor y dónde lo podemos encontrar? Como es de esperarse en un mundo de diversidad, la concepción de valor no es la excepción, es por ello que los pensadores han hecho la labor de crear dos corrientes que los definen, una objetivista, donde los valores son objetos, y la otra es subjetivista, donde los valores dependen de las personas. Según Max Scheler, Brentano, Husserl y Hartmann, los valores son cualidades independientes a los objetos (que son sus portadores) y a los fines (a donde apuntan) e inmutables, existen, sean captados o no, son absolutos en sí, inaccesibles a la razón, relacionados con una intencionalidad de un sujeto hacia un objeto, misma que no es intelectual, sino emocional y moral, lo que nos da una gradanción en la preferencia, existiendo además “axiomas” axiológicos (que no pueden ser demostrados), aún así, puede existir una racionalidad en el acto de valorar, como en el caso de la alegría racional al tener la certeza sobre la existencia de algo que se valora como positivo. Sirven a la vez como medida, su objetivismo puede ser captado al ser afectado o atrapado por el valor. Ahora, atendiendo al enfoque subjetivista, representado por Federico Nietsche (1884-1900), Alexius Meinong (1853-1921), Christian Von Ehrenfels (1850-1932) y Ralph Barton Perry (1875-1957) dicen que los valores son una creación de los hombres estabilizados temporalmente, cuyo cambio es necesario para el progreso humano, un objeto posee valor en tanto pueda suministrar una base afectiva, el valor produce agrado tanto por la existencia como inexistencia del objeto. La base para los valores radica en el apetito o deseo, que son quienes confieren valor a las cosas y cuyo interés consiste en la actitud afectivo-motora a favor o en contra del objeto, lo que le da el valor y no al revés. Es ese interés lo que refiere como agrado-desagrado, deseo-aversión, búsqueda-rechazo.
\section{Definición de laicidad.}
Aunque el ser laico no es un valor, sino una característica, y pareciera salir de contexto, es necesario saber qué es laico para fines del presente trabajo, así pues, siguiendo la misma mecánica, de acuerdo a la REA (2010): “laico, ca adj. No eclesiástico ni religioso, civil: misionero laico. También s.: los laicos colaboran con la Iglesia. [Escuela o enseñanza] que prescinde de la instrucción religiosa: colegio laico”.

Miguel Ángel Muñoz (2008), profesor de filosofía va definiendo al ser laico como una persona que es aconfesa o que no profesa una religión en particular. Más adelante retoma definiciones etimológicas y semánticas para definir laico como laicus, “que no tiene órdenes religiosas o que no pertenece al clero” y del término griego laos “pueblo, multitud indiferenciada”. Es precisamente de aquí de donde lo toma el Cristianismo para oponerlo a kleros “jerarquía eclesiástica, autoridad de la ekklesía”.

Es importante esta explicación, puesto que por mucha gente, el término laico se asocia directamente como “no ser católico”, mientras que para los cristianos (más globalmente) refiere a los que formamos parte de la Iglesia y que no pertenecemos a la jerarquía eclesiástica. Entonces, ser laico sería no exactamente “no ser católico”, sino simplemente “no ser religioso” o no pertenecer a una orden religiosa, sea la confesión o credo que sea. Y dentro del cristianismo, los laicos somos “el pueblo o multitud indiferenciado, carente de órdenes religiosas y que no pertenece al clero”.
\section{Diferencia entre Laico y Ateísmo.}
Durante un debate, celebrado el 22 de mayo de 2008, los profesores Esteban Cortijo, profesor de filosofía en Bachillerato y Presidente del Ateneo, autor de diversos estudios sobre la obra filosófica de Mario Roso de Luna e Isidoro Reguera, profesor de filosofía en la Universidad de Extremadura, dijeron que en este país no hemos sido laicos, aunque sí, en buena parte, anticlericales, que se explica por la presencia de un clericalismo agresivo y totalitarista. Se puede ser laico y no ser ateo, el laicismo, que es de sentido común, se puede decir que defiende la neutralidad del Estado ante las confesiones religiosas. Que dicha neutralidad permitirá la mejor convivencia entre las personas. Y que por lo tanto es un principio democrático incuestionable. El laicismo no es una cuestión religiosa, sino que atañe a lo político y lo social y que no supone una postura beligerante contra las iglesias, y recordaron como Ortega y Gasset en 1910 en una conferencia pronunciada en Bilbao defendía el laicismo y especialmente la escuela laica cuando dijo que hay que educar a la ciudad para educar al individuo y que la escuela laica debería ser prioritaria, una escuela laica instituida por el Estado, porque la sociedad es la única educadora y el fin de la educación es la sociedad.

La RAE (2010) define al Ateísmo como: “Sistema de ideas que niega la fe en lo sobrenatural (espíritus, dioses, vida de ultratumba, \&c.). El objeto del ateísmo es explicar las fuentes y causas del origen y existencia de la religión, criticar las creencias religiosas desde el punto de vista de la visión científica del mundo, aclarar el papel social de la religión, señalar de qué manera pueden superarse los prejuicios religiosos”.

En el Diccionario Enciclopédico Hispano-Americano (1887-1910) se menciona que el ateísmo es, primordialmente, una negación referente a la concepción de un dios y a su vez, el mismo diccionario cita a Proudhon y a madame Stael al decir que: «es menos lógico el ateísmo que la fe», «¿El ateísmo espiritualiza la materia o materializa el espíritu?», además que cita la exigencia de D'Alambert para distinguir «la ignorancia o desconocimiento de Dios» de «la posesión de su idea, que es después rechazada o negada» y que es a lo que refiere el ateísmo. Por último, J. Reynaud dice (y que resulta interesante): «se puede negar determinada concepción de la Divinidad, sin por ello negar la existencia de Dios. No lo entienden así los hombres intolerantes, para quienes no existe más Dios que su Dios (el que ellos conciben o dogmáticamente creen y confiesan), y para ellos oponerse a su creencia equivale a profesar el ateísmo. De esto resulta que no hay nombre más frecuentemente atribuido por los apóstoles de todas las religiones a sus adversarios que el de ateo».

Entonces, mientras que laico es la persona que es indiferenciable en el ámbito religioso, pero que no es carente de credo, sino que lo profesa, no importando cuál fuese y que además, el ser laico es una cuestión política y social, y que además no está en una postura contraria a las religiones, el ateo es aquel que niega la existencia de una divinidad, que busca el fundamento de la existencia de las religiones, sin embargo, no agrede a una religión particular, sino que toma una postura científica, inquisitiva, investigadora que lo lleva a buscar una verdad.
\section{Educación laica.}
La Constitución Política de los Estados Unidos Mexicanos, en su artículo 3° (1993) expresa que todo individuo tiene derecho a la educación, y que la misma, impartida por el Estado tenderá a “desarrollar armónicamente todas las facultades del ser humano y fomentar en el, a la vez, el amor a la patria y la conciencia de la solidaridad internacional, en la independencia y en la justicia… Garantizada por el Artículo 24°, dicha educación será laica y por tanto, se mantendrá por completo ajena a cualquier doctrina religiosa… esa educación se basará en los resultados del progreso científico, luchará contra la ignorancia y sus efectos, las servidumbres, los fanatismos y los prejuicios… Será democrática, considerando la democracia… como un sistema de vida fundado en el constante mejoramiento económico, social y cultural del pueblo…” Esto define a la educación como independiente de credos religiosos, lo cual coincide con la definición de laicidad que la RAE expresa y que el profesor Miguel Ángel Muñoz menciona, además de que expresa claramente “se mantendrá ajena”, es decir, ni estará a favor de religión alguna ni en su contra y sí reconocerá la existencia de las mismas, lo cual los confiere en un Estado y una Educación “no ateos”, un Estado y una Educación basados en el respeto de las personas y el reconocimiento de sus creencias. Más aún, el mismo artículo hace mención de otra característica: la democracia, a la cual define como “sistema de vida fundado en el constante mejoramiento económico, social y cultural del pueblo”. Es por tanto, y de acuerdo a nuestra Constitución, algo deseable el alcanzar la democracia. Pero, aparte de definirse como un sistema de vida… ¿Qué es la democracia?

\section{Democracia, ¿qué es y cómo se relaciona con los valores y la laicidad?}
Nuevamente, la RAE (2010) define a la democracia como “Doctrina política favorable a la intervención del pueblo en el gobierno” y “Predominio del pueblo en el gobierno político de un Estado”. Etimológicamente hablando, sabemos que el término democracia proviene del antiguo griego ($\delta\eta\mu o \kappa\rho\alpha\tau\iota\alpha$)a partir de los vocablos $\delta\eta\mu o\varsigma$ (<<demos>>, que puede traducirse como <<pueblo>>) y $K\rho\alpha\tau o\varsigma$ (<<Kratos>> que puede traducirse como <<poder>> o <<gobierno>>). Aún así, esta significación no es suficiente actualmente, ya que implica la imposición de lo que la mayoría dicta. Hoy en día, en los gobiernos demócratas se busca un equilibro de respeto y cordialidad donde las decisiones que tome la mayoría incluya de la mejor manera lo que la minoría busca, esto mediante mecanismos que garanticen su consecución. Se han planteado diferentes formas de democracia: directa, en la que la totalidad de la comunidad toma decisiones referentes a asuntos que afectan a todos, indirecta, donde el pueblo elige representantes que tomarán decisiones en beneficio de sus representados y mixta, donde los representantes realizan el trabajo del consenso y la decisión final en ciertos aspectos la toma el pueblo.

Es interesante cómo, pese a que la democracia es algo deseable, alcanzado en diferentes grados en diversos países, de acuerdo al informe “El desarrollo de la democracia en América Latina”, publicado en fechas posteriores al 2004, el 54.7\% de los latinoamericanos estaban dispuestos a aceptar un gobierno autoritario, siempre que éste pudiera resolver la situación económica, y esto se refleja en la pérdida de confianza en los partidos políticos, en el bajo crecimiento del PIB per cápita (3856 dólares en AL frente a 36,100 en EUA para el año 2003), aunque los niveles de pobreza, en forma relativa reflejaron una disminución, en términos absolutos, aumentó la cantidad de personas que se ubicaban por debajo de la línea de pobreza, pasando de 190 millones de personas en 1998 a 209 millones en el año 2001. Esto nos hace pensar en que debe haber algo más en la democracia que la simple elección o toma de decisiones que beneficien a la mayoría sin agraviar a las minorías, algo debe estar faltando o en algo debemos estar errando el rumbo.

Interesante también lo que el profesor Isidoro Reguera, citando a Wittgenstein recordó que el laicismo, que es de sentido común, se puede decir: Que defiende la neutralidad del Estado ante las confesiones religiosas. Que dicha neutralidad permitirá la mejor convivencia entre las personas. Y que por lo tanto es un principio democrático incuestionable. El laicismo no es una cuestión religiosa, sino que atañe a lo político y lo social y que no supone una postura beligerante contra las iglesias. Interesante porque muestra una relación entre valores, laicidad, religión y democracia, su relación muestra a la democracia como algo deseable profundamente, para beneficio de los grupos sociales, lo que la constituye en un valor, sin embargo, la democracia por sí sola, nos ha mostrado ser insuficiente para la mejora de los pueblos, y América Latina es un claro ejemplo, tal como los datos que expuse líneas antes nos lo muestra.

Entonces ¿Qué falta en la democracia? ¿Qué falla en nuestra educación? ¿Qué es lo que necesitamos para que la democracia sea algo efectivo como ha sido en otras naciones? Si hablamos de democracia como el sistema de vida incluyente que busca hacer cumplir la voluntad de las mayorías sin excluir o agraviar a las minorías, entonces hablamos de respeto, de tolerancia, si hablamos de democracia como forma de gobierno representativo donde un grupo elegido por la comunidad, delibera a favor de sus representados, entonces hablamos de comunicación (representantes-representados), de honestidad (lo que se diga sea la verdad), de empatía (comprender la situación de los representados), nuevamente de respeto y tolerancia, en cualquiera de estos dos enfoques, implica comprender que el beneficio propio debe contribuir al beneficio colectivo y el aumento de beneficio colectivo conduce al incremento del beneficio propio.

Tolerancia, Honestidad, Empatía, Respeto, Democracia, en sí mismos son cosas deseables, que benefician a la comunidad, coadyuvados por y coadyuvantes de la comunicación. ¡Momento!, si estos son elementos deseables para beneficio colectivo y personal, existentes en sí mismos y que pueden ser percibidos en mayor o menor medida, aplicados a otros elementos (comunidad, persona, organización, etc.), entonces nos encontramos ante la existencia y búsqueda de valores mediante la aplicación y cumplimiento de leyes en un estado que es independiente de cualquier credo religioso y que a su vez los respeta sin coartar su proceder ni imponer alguno en particular y por tanto es LAICO.

\section{Conclusión.}
Si nos encontramos ante un estado laico, que refleja en sus leyes el deseo de mejorar y crecer, de regular el comportamiento ciudadano sin importar creencia, sin coartarlas ni imponerlas, de regirse por medio de un sistema democrático (con todo lo que ello implica), entonces nos encontramos ante la situación de que es necesario plantear la forma de alcanzar aquello que es deseable para beneficio de todos, para ello es necesario enseñar a buscar y practicar aquello que nos ayude a crecer en forma positiva, aquello que llamamos VALORES, como lo son la tolerancia, el respeto, la empatía, la honestidad. Es curioso como en realidad, aquello que se busca como Estado laico, también se busca en cuanto a credos y religiones, ambos buscan en realidad el crecimiento de sus comunidades que son la misma: la comunidad de seres humanos. Es interesante ver esta extraña y a la vez obvia simbiosis que nos negamos muchas veces a ver, que es necesaria para el buen crecimiento de las personas, puesto que las personas somos más que los individuos que vemos, es interesante comprender que si bien se busca un Estado democrático laico, éste no tiene que ser ateo ni inhibidor de las prácticas piadosas y muestras de fe que cada persona profese. Si tuviéramos la disposición a escuchar lo que la otra parte tiene que decir, descubriríamos que son más los puntos que tenemos en común que los que nos diferencian y veríamos que lo que buscamos por estos medios es exactamente lo mismo: el bien común y la trascendencia.

Entonces ¿qué concluimos? Concluimos que si bien el Estado tiene la obligación de educar en valores a fin de lograr una democracia plena que lleve al crecimiento de las naciones, no debería impedir, coartar ni obstaculizar el accionar de ninguna de las organizaciones religiosas, en particular el de aquellas cuyos fines son análogos a los que son deseables para el crecimiento del colectivo (“No se lo impidáis, pues el que no está contra vosotros, está por vosotros” Lc. 9:50).

\section{Referencias}
Biblia de Jerusalén (1975). Evangelio de san Lucas. Bilbao. Editorial Española Desclée de Brouwer
Cáceres Laica(2008). Laicismo y Religión, una visión desde la Filosofía. Recuperado el 27 de octubre de 2010 de \url{http://cacereslaica.wordpress.com/2008/05/23/laicismo-y-religion-una-vision-desde-la-filosofia/}
Diccionario de la lengua española © 2005 Espasa-Calpe.
Diccionario enciclopédico hispano-americano (1887-1910). Recuperado el 26 de octubre de 2010 de \url{http://www.e-torredebabel.com/Enciclopedia-Hispano-Americana/V2/ateismo-filosofia-D-E-H-A.htm}
Diccionario Soviético de Filosofía (1965). Recuperado el 30 de octubre de 2010 de \url{http://www.filosofia.org/enc/ros/ateismo.htm}
Gerardo Remolina Vargas (2005). La formación en valores.
La democracia en América Latina. El desarrollo de la democracia en América Latina (2005).
Miguel Ángel Muñoz, Laicismo día tras día Fundamentación filosófico-política del laicismo, recuperado el 2 de noviembre de 2010 de \url{http://www.europalaica.com/colaboraciones/c070524_Laicismo_dia_tras_dia.pdf}
Pablo Latapi. (2001). “Valores y educación” Ingenierías Vol. IV, No. 11, 59-69
Wikipedia (2010). Laico (Religioso). Recuperado el 30 de octubre de 2010 de \url{http://es.wikipedia.org/wiki/Laico_%28religioso%29}
