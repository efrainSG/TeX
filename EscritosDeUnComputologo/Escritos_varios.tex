\chapter{Al rescate de nuestras tradiciones}
\section{Parte 1}
Como sabemos, las tradiciones son parte de la identidad de un pueblo o un pais, solo que en un mundo tan globalizado, las tradiciones se van reemplazando por las costumbres del pais con más empuje comercial. Es por ello que me estoy dedicando a postear lo que son las bellas tradiciones de México; para que no se pierdan pongo mi granito de arena en este blog.

Las ofrendas tienen su historia muy antigua y las tradicionales son una mezcla de las culturas prehispánicas e hibéricas.

Contienen elementos muy interesantes y cada uno con un significado. Aquí pongo unas fotos de algunas ofrendas y más abajo el significado de algunos elementos tradicionales.

\section{Parte 2}
Diciembre... qué bonito mes... Iniciando con la fiesta del el 12 en que se celebra a la Virgen de Guadalupe y muchísima gente va en procesión a visitarla a la Basílica de Guadalupe en México DF (también conocida como "La Villita") y siguiendo con "las posadas" (de lo que trata este post)

Pues las Posadas son una tradición mexicana donde durante 9 días se recuerda el viaje de María y José desde Nazaret hasta Belén, no porque haya sido en 9 días.

Origen de la tradición:

Las posadas son fiestas que tienen como fin, preparar la Navidad. Comienzan el día 16 y terminan el día 24 de Diciembre.

Los misioneros españoles que llegaron a México a finales del siglo XVI, aprovecharon estas costumbres religiosas para inculcar en los indígenas el espíritu evangélico y dieron a las fiestas aztecas un sentido cristianos, lo que serviría como preparación para recibir a Jesús en su corazón el día de Navidad.

En 1587 el superior del convento de San Agustín de Acolman, Fray Diego de Soria, obtuvo del Papa Sixto V, un permiso que autorizaba en la nueva España la celebración de unas Misas llamadas "de aguinaldos" del 16 al 24 de diciembre. En estas Misas, se intercalaban pasajes y escenas de la Navidad. Para hacerlas más atractivas y amenas, se les agregaron luces de bengala, cohetes y villancicos y posteriormente, la piñata.

En San Agustín de Acolman, con los misioneros agustinos, fue donde tuvieron origen las posadas.

Los misioneros convocaban al pueblo al atrio de las iglesias y conventos y ahí rezaban una novena, que se iniciaba con el rezo del Santo Rosario, acompañada de cantos y representaciones basadas en el Evangelio, como recordatorio de la espera del Niño y del peregrinar de José y María de Nazaret a Belén para empadronarse. Las posadas se llevaban a cabo los nueve días previos a la Navidad, simbolizando los nueve meses de espera de María. Al terminar, los monjes repartían a los asistentes fruta y dulces como signo de las gracias que recibían aquellos que aceptaban la doctrina de Jesús.

Las posadas, con el tiempo, se comenzaron a llevar a cabo en barrios y en casas, pasando a la vida familiar. Estas comienzan con el rezo del Rosario y el canto de las letanías. Durante el canto, los asistentes forman dos filas que terminan con 2 niños que llevan unas imágenes de la Santísima Virgen y de San José: los peregrinos que iban a Belén. Al terminar las letanías se dividen en dos grupos: uno entra a la casa y otro pide posada imitando a San José y la Santísima Virgen cuando llegaron a Belén. Los peregrinos reciben acogida por parte del grupo que se encuentra en el interior. Luego sigue la fiesta con el canto de villancicos y se termina rompiendo las piñatas y distribuyendo los "aguinaldos".

Significado de la tradición:

Las posadas son un medio para preparar con alegría y oración nuestro corazón para la venida de Jesucristo, y para recordar y vivir los momentos que pasaron José y María antes del Nacimiento de Jesús.

Algo que no debes olvidar

Debemos vivir las tradiciones y costumbres navideñas con el significado interior y no sólo el exterior para preparar nuestro corazón para el nacimiento de Jesús.

Cuida tu fe

Algunas personas te podrán decir que estas costumbres y tradiciones las ha inventado la gente para divertirse y los comercios para vender. Recuerda que hay mucho significado detrás de cada una y trata de vivir estas tradiciones con el sentido profundo que tienen. Así, el 24 de diciembre no solo será un festejo más, sino que habrás preparado tu corazón con un verdadero amor a Dios y a tu prójimo.

\section{Parte 3}
Las piñatas tienen su origen en China, donde, al inicio del año chino en primavera, se llevaba a cabo una ceremonia en la cual los chinos elaboraban con papel la figura de un buey, la cubrían con papeles de colores y le colgaban herramientas agrícolas.

Los colores de la figura simbolizaban las condiciones en que se desarrollaría el año con respecto a la agricultura. Se rellenaban con cinco clases de semillas que caían cuando los reyes mandarines le pegaban a la piñata con varas de diferentes colores. Cuando ya estaba vacía, se quemaba y la gente trataba de obtener parte de las cenizas, pues consideraban que daba buena suerte para todo el año.

Esta costumbre china llegó a Europa y en Italia fue donde le dieron un sentido religioso. Primero las utilizaron para las fiestas de Cuaresma, que concuerdan con el inicio de la primavera.

La piñata está hecha con una olla de barro cubierta con papel de colores brillantes y representa al demonio, que suele presentar al mal como algo llamativo para que cautive al hombre y ceda a la tentación.

La piñata clásica es como una estrella de siete picos que representan a los siete pecados capitales: soberbia, avaricia, lujuria, ira, gula, envidia y pereza.

Pegarle a la piñata con los ojos vendados representa la fe, virtud que nos permite creer sin tener que ver.

El palo con el que se le pega a la piñata representa a la fuerza de la virtud que rompe con los falsos y engañosos deleites del mundo. Las virtudes que hay que cultivar para vencer los pecados capitales son: contra la soberbia, la humildad; contra la avaricia, la magnanimidad; contra la ira, la paciencia; contra la envidia, la generosidad; contra la lujuria, la castidad; contra la gula, la templanza; contra la pereza, la diligencia. Con la ayuda de Dios, se destruye al mal y así se descubren los frutos que hay dentro de la piñata , que representan a las gracias de Dios.

El relleno de la piñata es símbolo del amor de Dios porque al romper con el mal, se obtienen los bienes anhelados.

De Italia, la costumbre de romper piñatas en Cuaresma llegó a España. Los españoles instauraron una fiesta cada primer domingo de Cuaresma y la llamaron "El baile de la piñata".

Romper la piñata al inicio de la Cuaresma simbolizaba el deseo de acabar con el mal en la propia vida, de convertir el corazón para volver a Dios y de recibir los bienes eternos.

A principios del siglo XVI esta tradición era desconocida en América. Sin embargo en México, los indios mayas, que gustaban mucho del deporte, tenían un juego en el que trataban de romper con los ojos vendados una olla de barro llena de chocolate que se balanceaba detenida de una cuerda. A los frailes evangelizadores se les ocurrió que serviría de catequesis dar un sentido religioso al juego de la olla, enseñándoles el significado religioso de las piñatas y promoviendo que se rompieran durante el tiempo de Adviento como un complemento a las fiestas de las Posadas y con el mismo sentido de conversión que le daban los europeos.

Los "aguinaldos" son bolsitas o canastas con dulces y galletas que se les entregan a las personas que, por su edad o por su salud, no pueden acercarse a recoger los dulces y las frutas de las piñatas, con la idea de que nadie se quede sin recibir los beneficios de la piñata y sin participar de la alegría de la fiesta.
Al repartir los aguinaldos debemos pensar en que todos llevamos dentro nuestro propio "relleno", es decir, una serie de cualidades, de dones que debemos descubrir y desarrollar para compartir con los demás. Cuando compartimos, nos llenamos de felicidad tanto al dar como al recibir.

\section{Parte 4}
¿Cómo hacer piñatas?

La fabricación de piñatas es una hermosa tradición que se ha ido perdiendo por la vertiginosidad de nuestras actividades actuales.

Me preguntaron cómo hacer unas piñatas y pues aquí le spongo cómo hacer una bella piñata tradicional de estrella de 7 picos.

Necesitaremos:

\begin{itemize}
\item Olla de barro sin asas y sin pintar, es decir 100\% barro
\item Papel periódico (el que se hace de papel revolución, te venden un periódico como de 1 kilo de peso por unos pesos y si es ya pasado, pues más barato)
\item Harina (un medio kilo)
\item Agua
\item Limón
\item Papel de china de colores
\item Tijeras
\item Palo de escoba
\end{itemize}

\begin{enumerate}
\item Ponemos una olla con medio litro de agua a calentar a fuego medio y de a pocos agregamos la harina y vamos moviendo para que no se pegue y los grumos los vamos desbaratando, le agregamos el jugo de medio limón y seguimos moviendo hasta que obtengamos una pasta espesa, de consistencia similar a los pegamentos de contacto.
\item Ponemos la olla de barro para la piñata sobre una mesa con la "boca" hacia abajo, rasgamos el periódico en tiras o fragmentos de tamaño medio (de unos 20 a 25 cm por lado, nada de cortarlo con tijeras, lo bonito es el rasgarlo a mano), los embadurnamos con el engrudo que hicimos y lo pegamos sobre la olla, así hasta cubrirla totalmente, pueden ser 2 o 3 capas, según el gusto.
\item Con hojas de periódico hacemos "cucuruchos" o conos para formar los picos de la estrella, recordemos que deben ser 7
\item A cada cucurucho le hacemos muescas en la base para poder pegarlos al cuerpo de la piñata y para reforzar, cubrimos las pestañas con tiras de periódico con engrudo
\item Comenzamos por hacer con papel de china los flecos que adornan las puntas de la estrella, para ello tomamos un pliego de papel de china y lo cortamos en tiras de unos 30 cm de ancho, luego hacemos unos cortes paralelos entre sí y perpendiculares al largo del papel; los cortes deben ser de aproximadamente 2/3 del ancho del papel, luego embadurnamos con el engrudo la parte sin cortes y cubrimos una de las puntas de la estrella y así para los siete picos.
\item Los otros pliegos de papel de china los cortamos a lo largo en tiras de unos 15 cm de ancho; a cada tira le embarramos engrudo por una orilla siguiendo el largo de la tira, unos 3 cm, ponemos el palo de escoba al centro de la tira sin que toque el engrudo y luego le pasamos encima el otro extremo del papel y pegamos ambas orillas como haciendo una canastilla o columpio para el palo; y así para cada tira, ya que haya secado lo suficiente, hacemos cortes paralelos entre sí y perpendiculares al largo de la tira sin que los cortes lleguen a la parte pegada.
"Vestimos" la piñata con los "chinitos" que hicimos, comenzando desde los extremos de picos de la estrella y hacia el cuerpo cuidando que las orillas sin corte sean las que embadurnemos y queden colocadas hacia el centro. El cuerpo de la piñata lo cubrimos desde la base y hacia la "boca".
\end{enumerate}
\section{Parte 5}
El belén, también llamado nacimiento, pesebre, portal o pasitos en los diferentes países y regiones de habla hispana, es la representación plástica de escenas de la Natividad de Jesús de Nazaret, que se suele exponer en las iglesias y en los hogares. La construcción y exhibición de belenes forma parte de la liturgia navideña en muchas partes del mundo, especialmente en la tradición católica. En la plaza de San Pedro, en el Vaticano, se arma anualmente un belén de tamaño natural.

En 1223 san Francisco de Asís dio origen a los pesebres o nacimientos, en una ermita de Greccio. Pero en un principio, la escena del nacimiento de Cristo era representado por personas reales dentro de un establo con animales, no con figuras de cerámica o barro.

En este primer nacimiento, san Francisco ya incluía al buey y al asno, basándose en la lectura de Isaías: "Conoce el buey a su dueño, y el asno el pesebre de su amo. Israel no conoce, mi pueblo no discierne" (Is. 1,3). Aunque estos animales ya aparecen en el "Pesebre" del siglo IV, descubierto en las catacumbas de San Sebastián en 1877.

Posteriormente en el siglo XIV, la idea de los nacimientos se consolidó como tradición en la península itálica. En Nápoles, el rey Carlos III promovió la difusión de los nacimientos en España. Aunque los frailes franciscanos ya empezaban a difundirlos desde el siglo XIII, al igual que en Alemania.

Con las modas renacentista y barroca, la decoración de los nacimientos cobró fuerza y se volvió un arte. En América, los franciscanos usaron los belenes como método de evangelización. Fue allí donde comenzaron a ser anacrónicos, ya que incluían animales y plantas americanas, que en Palestina no se conocían en tiempos de Jesús, como los guajolotes, magueyes y nopales.

\chapter{¡Lleve sus calaveritas vaciladoras!}
Contaré una pequeña historia a como la recuerdo, espero que las fuentes de donde la obtuve puedan corregirme si en algo yerro.

Hace tiempo, vivían en una vecindad una familia integrada por el papá, la mamá y sus hijos (recuerdo que en ese momento eran seis). El papá era trabajador de Teléfonos de México, de los que salían a trabajar en su bicicletita con su cinto con herramientas. Resulta que por esas fechas llegaba a su casa en su bicicleta cargada con unos cartones y un poco de barro.

Entonces se dedicaba a recortar ese cartón en forma de pequeñas cajas con una ventanita en uno de los lados. Con el barro daba forma a las pequeñas figuras que luego ponía a secar y después cocía a fuego, para pintarlas de blanco, entonces las colocaba dentro de las cajitas, les ponía una tapa, un hilo, las pintaba de negro y con un ingenioso y sencillo mecanismo les daba vida.

Y salía a venderlas a la voz de "¡Lleve sus calaveritas vaciladoras!", un curioso juguete de temporada que con un hilo hacía que se levantara del pequeño ataud la simpática figura de un esqueletito con los brazos cruzados sobre el pecho, como si se sentara repentinamente, y con un leve movimiento, y por acción de la gravedad, la calaverita volvía a acostarse en su caja y la tapa se cerraba al momento.

Así fue por mucho tiempo, cada año vendía sus calaveritas vaciladoras hasta que ya no necesitó hacerlo más, no porque se haya vuelto rico, sino porque sus hijos habían crecido, y las cosas iban progresando.

Yo conocí esas figuritas de manos de mi abuelito y luego que mi papá nos mostró cómo se hacían.

Sí, fue mi abuelito Silvestre Serna Martínez quien hizo el simpático juguete, él construyó y refinó sus moldes de barro para hacerlas. Al pasar de los años, comencé a ver en los mercados y tiánguis unas calaveritas similares, hechas de azucar y un poquito más toscas. Ya no las he vuelto a ver, pero nunca me olvido por estas fechas de que yo también tuve mi calaverita vaciladora.

\chapter{Proceso de Investigación}
\section{Parte 1}
Es interesante cómo se realiza un proceso investigativo, es un conjunto de actividades aplicables a muchos ámbitos de la vida, no solo académicos (realizo entrevistas leo, busco, infiero, recopilo, organizo, demuestro y concluyo). Por ejemplo, si yo quiero realizar un proyecto y tal vez solo tenga la idea de "quiero hacer un proyecto", puedo comenzar por tomar ideas de mi entorno, de ahí, puedo plantearme una serie de preguntas que me llevarán a encontrar el punto clave: "qué es lo que voy a hacer".

Siguendo luego por... buscar artículos, noticias, reportajes, investigaciones relativas a eso que quiero hacer (claro que hay que buscar fuentes válidas, fidedignas y no cualquier nota pegada en alguna página).

Luego, ya que tengo mis preguntas iniciales y una idea de lo que quiero hacer, me dedico a realizan entrevistas exploratorias, algo así como para tantear el terreno. Estas entrevistas son con gente experta en el tema que involucra lo que quiero hacer, pero eso no quiere decir que nos van a decir cómo hacer el trabajo o que no tiene sentido o nos fusilen o madruguen la idea.

Defino ahora una serie de marcos para mi proyecto, uno teórico donde recopile información existente y antecedentes relacionados, uno conceptual donde sirva para aclararme y aclarar a los demás cada término involucrado en lo que quiero de proyecto, y un marco contextual, es decir, en qué ambiente realizaré y funcionará mi trabajo.

Paso seguido: plantear la hipótesis (si es que requerimos de una), recopilar información cualitativa y/o cuantitativa, según lo que requiera mi trabajo.

\section{Parte 2}
Si... es complejo realizar un proceso de investigación... plantearte una pregunta sobre lo que vas a realizar (incluyendo proyectos), buscar fuentes de información relativas a ese trabajo para saber en qué van las cosas, luego que lees un tanto comienzas a ver el lugar para el que se va a realizar dicho proyecto (de investigación o de otro tipo), entonces te das cuenta... ¡la pregunta es MUY amplia hay que "recortarla"!... piensas un poco (o un mucho) ¿cómo la recorto?... la pregunta debe ser clara, precisa, concisa y maciza :p... piensas y ¡LISTO! encontré la pregunta ideal, es clara, precisa, concisa y maciza :D ahora sí puedo trabajar... ¡MOMENTO! no expliqué en dónde trabajaré, así como se ve es poco factible... :( a reacomodar... ¿la pregunta explica lo que voy a hacer, dónde, cómo y cuando? ¿es breve, clara, precisa y factible? ¡ADELANTE! a comenzar el trabajo.

Ya definí el marco contextual (dónde y cuándo), el conceptual (a qué me refiero con cada palabrita de mi pregunta inicial y las que se relacionan directamente), el teórico (los trabajos que ya existen en el campo de mi investigación y que me pueden servir de base). Tengo mi objetivo general y que he desglozado en objetivos específicos, también mi hipótesis está definida (si es que la necesito), un planteamiento del problema y una justificación de por qué haré mi investigación.

\section{Proyecto de investigación/evaluación}
La Universidad Hispana de Puebla enfrenta un grave problema en las carreras de Computación, Electrónica y Sistemas: El número de estudiantes en los grupos se van reduciendo al paso del tiempo, ingresan cada vez menos, se mantienen cada vez menos y terminan aún menos. Esto amenaza con provocar la desaparición de dichas carreras.
El problema de la deserción puede desembocar en el cierre de las carreras en cuestión, ya que los costos en instrucción y mantenimiento de instalaciones y mobiliario, en comparación con el número de estudiantes que los aprovechan, toma una proporción negativa que lo vuelve poco o nada sustentable, y este fenómeno no es exclusivo de las carreras cuyo interés compete al presente estudio, sino que es extensivo a las demás carreras de la Universidad, si es que el fenómeno de deserción se propaga hacia ellas.

Al cerrarse las carreras en la Universidad, se pierden opciones de estudio en el área para los estudiantes que egresan del nivel medio superior y los que concluyen sus estudios de nivel técnico superior, por lo que optan por segundas opciones que no son del todo satisfactorias para sus intereses, incorporarse a la vida laboral con oportunidades limitadas para su nivel máximo de estudios, tomar cursos técnicos o simplemente no seguir estudiando en espera de una nueva oportunidad en instituciones con una demanda de matrículas mayor.

Al reducirse la cantidad de alumnos en las carreras los costos por pago a catedráticos (cursos de actualización, pago de salarios, prestaciones) se vuelven insostenibles, con lo que las opciones son reducción en sus percepciones económicas (en algunos casos puede llevar a que el catedrático opte por otra opción de trabajo) o el recorte de personal (desaprovechando el potencial de los catedráticos para transmitir conocimiento y formar profesionistas de calidad)...

Esta es parte de la introducción a mi trabajo de Evaluación que permita visualizar el escenario dentro del cual, la Universidad se desenvuelve. Seguro estoy que no es la únia Universidad en una situación de este tipo, por ello, es posible tomar las ideas y procesos que se plasman en el documento y aplicarlos a evaluaciones en otras instituciones.

\chapter{Un largo y sinuoso camino}
\section{Hacia la comprensión de la investigación / Evaluación}

Así como el título e la canción e Lennon y Mccartney "the long and winding road", nosotros también hemos recorrido un largo y sinuoso camino durante todo el módulo de investigación pero especialmente en el desarrollo del Taller 2. Pero antes de repasar brevemente lo que hemos vivido y aprendido hay que presentarse y decir que a pesar de llevar unos meses conociéndonos desde el módulo propedéutico es hasta ahora cuando realmente hemos establecido una gran relación de trabajo y puede decirse que también de amistad, incluso hemos desarrollado una comunicación más frecuente con nuestra estimada tutora, la Mtra. Carmen Navarro y otros compañeros de otras sedes. Viridiana Xilotl, Dorian Ruíz, Valerie Velasco y yo pertenecemos al grupo 1 del módulo de Investigación y estamos en la sede BUAP.

Desde que comenzó el módulo, todos hemos asistido sin falta cada miércoles a nuestra sede para compartir dudas, experiencias  e inquietudes que han surgido de las actividades que hemos realizado.
Coincidimos en que hemos experimentado un crecimiento como investigadores porque es hasta ahora cuando comprendemos el verdadero valor del proceso sistemático de la investigación/evaluación de las Ciencias Sociales.

Gracias a las actividades, telesesiones y ayuda de nuestra tutora hemos logrado aprender y experimentar qué es una investigación/evaluación y las complicaciones que podemos llegar a tener durante su desarrollo.

Descubrimos que la parte principal de este proceso es la pregunta de investigación que se formula a partir de inquietudes que tenemos del mundo que nos rodea, pudimos ver que si no está bien hecha desde el inicio provocará que el proceso de estudio sea largo y sinuoso y que probablemente no obtengamos los resultados esperados.
