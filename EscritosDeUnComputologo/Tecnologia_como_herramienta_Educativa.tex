La tecnología como herramienta en el desarrollo de la educación.
Aprovechar la tecnología en el entretenimiento a favor de la educación.
Educación y Tecnología, la brecha y la simbiosis.

Cada quien proviene de una generación diferente, es algo que en la última reunión familiar estuvimos comentando, platicábamos acerca de cómo han ido cambiando las cosas y que algunas nunca fueron vistas por los más jóvenes, los que pertenecemos a una generación intermedia apenas alcanzamos a ver, mientras que nuestros padres y abuelos las vivieron plenamente. Comentábamos sobre los programas, la forma de divertirnos y hasta la manera en que (a manera de anécdota) en ocasiones llegaban a prepararse, si se puede llamar de esa forma, para exámenes viendo películas en las salas de cine, tales como “los 300 héroes”, sabiendo que el profesor les solicitaría relatar un suceso histórico en forma oral, lo que se reducía entonces a platicar la película. Por otra parte hemos vivido el desarrollo de buena parte de las tecnologías informáticas aplicadas al entretenimiento, desde la creación de rudimentarios juegos de video, donde bastaba mover una perilla para que un rectángulo visto en el televisor subiera o bajara, hasta los más novedosos juegos que recrean las condiciones de un mundo, ahora llamado virtual donde se desarrollan épicos enfrentamientos, una modernización del juego de ajedrez que lo extiende a la interacción con más de un jugador, o combates donde intervienen ambientes, equipamiento y cualidades de los personajes. Esto sin mencionar la creciente complejidad en los controles de los mismos (auténticos teclados que se controlan con ambas manos a la vez).

Por el contrario, tal como expone Papert en su obra “La máquina de los niños”, la educación formal al interior de las aulas, en comparación a los avances en otras áreas, tales como el entretenimiento o la medicina, no ha recibido el mismo empuje para su evolución; lo cual, nos mantiene muchas veces en una mera relación expositor-receptor cuyo apoyo es la repetición y la lectura guiada más que el verdadero desarrollo y construcción de conocimiento.

La realidad de una educación tan rígida como ha sido en muchos lugares hasta ahora, es lo que a su vez provoca en los estudiantes los sentimientos de rechazo, frustración, aburrimiento y tedio por no encontrar la adecuada relación entre el conocimiento y capacidades que se buscan desarrollar en las horas de clase y tareas con las actividades cotidianas que cada individuo realiza. Mientras que en las actividades de ocio que se fueron mencionando al inicio, a la par de desarrollar ciertos procesos de análisis, abstracción, retención e incluso de repetición, se establece la relación acierto-premio o simplemente una distracción voluntaria de las obligaciones cotidianas.

Este contraste tan marcado es lo que lleva de manera constante a la búsqueda de técnicas eficaces para el aprendizaje, así como el uso de tecnologías multimedia que permitan la recepción de información a través de más de un canal. Es importante recordar que, si bien, los medios de comunicación que permiten divulgar información han existido desde hace ya mucho tiempo, es ahora que han cobrado una aceleración y un crecimiento exponenciales. La existencia de redes sociales, blogs, foros, wikis, salas de chat, redes P2P (que permiten compartir grandes cantidades de archivos entre personas), tienen una importancia tan marcada que es imposible pasarlos desapercibidos. Estas mismas herramientas, con un adecuado manejo pueden incentivar a la generación de conocimiento nuevo, reafirmar el existente, e incluso, acercar a las personas, en lugar de alejarlas, como se ha planteado muchas veces.

Por ende, el aprendizaje, que podrían realizar aquellas personas encargadas de la transmisión del conocimiento, sobre el uso de las tecnologías de la información y comunicación y las posibilidades que ofrecen, así como el desarrollo de capacidades de inventiva y abstracción, podrían conducir a la generación de nuevas herramientas que reduzcan ese tedio que representa el aprendizaje dentro de las aulas para extenderlo hacia la investigación independiente por parte de los alumnos.

Y bien, ¿Qué hay de aquel conocimiento que hemos puesto como base y que por mucho tiempo hemos requerido de repeticiones constantes a fin de hacerlo parte de nuestras mentes? ¿Cuáles podrían ser las estrategias? ¿Será posible cambiar la forma de enseñarlo? Sabemos, y es parte de la naturaleza, que la realización constante de una tarea se convierte en hábito, costumbre y hasta en ley. Sabemos de la existencia de hábitos buenos y malos, hábitos fáciles y difíciles de adquirir o de erradicar, según la circunstancia. Aún en ello se puede utilizar la tecnología a favor nuestro para afirmar un conocimiento. Es común, en el entretenimiento de los juegos que ciertas acciones conllevan un premio, ya sea individual o colectivo, y que entre más veces logremos realizar dicha acción mayor será la recompensa (mejor condición, mayor fuerza, velocidad, agilidad, destreza y reconocimiento por parte de quienes nos rodean). Es por ello que nos esforzamos en realizar dicha tarea, mientras que por el contrario, realizar otras acciones pueden llevar a penalizaciones o simplemente a no obtener premio alguno, con lo que sabemos que esas acciones no las debemos realizar o debemos evitarlas lo más posible. Esta repetición de acciones que llevan a un premio y evasión de acciones que no aportan beneficio conduce a la obtención de hábitos, y si esos hábitos se logran relacionar con conocimientos, el beneficio sería mayor, facilitando así la obtención y construcción de nuevo conocimiento por parte de las personas.

Pues bien, la forma en que hemos de aprovechar el crecimiento tecnológico para impulsar el aprendizaje, depende, en primera instancia, de que conozcamos las variedades de uso de la misma tecnología, en segundo término, depende de la inventiva que podemos desarrollar para hacer uso de las herramientas, y finalmente, depende de nuestra capacidad para transmitir el interés en su uso para adquirir conocimiento relacionándolo a situaciones reales donde se aprecie la existencia de un beneficio a corto plazo y que resulte en otro beneficio mayor, a mediano y largo plazo. De nosotros depende que la brecha expuesta por Papert entre la tecnología y la enseñanza en las aulas se reduzca hasta convertirla en esa simbiosis donde el conocimiento genera tecnología útil para desarrollar nuevo conocimiento.
