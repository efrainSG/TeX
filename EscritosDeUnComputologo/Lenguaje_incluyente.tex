Lenguaje Incluyente o no
Prólogo
Siempre hay polémicas y discusiones, desacuerdos, debates. Esto es lógico, ya que cada uno tiene un punto de vista diferente a consecuencia de lo que ve, lee, estudia, piensa, deduce.

¿Por qué lo digo? Porque recientemente me he metido en ese tipo de discusiones, específicamente la cuestión relacionada con la legalización (o no) del aborto. También he visto mucho en publicaciones lo relacionado con la llamada Equidad de Género, así como también la lucha por los derechos del colectivo LGBT+.

De esto último me ha llamado de forma particular la atención el hecho de luchar por un lenguaje incluyente.

Me gusta debatir y discutir por el hecho de manejar los argumentos, entender qué me falta por aprender. Pero desafortunadamente un debate implica que hay un ganador y un perdedor, que en ocasiones cada uno de los contendientes piensa que ganó. Por ello mejor escribo, de esta forma se pueden organizar mejor las ideas y quien lee puede analizar más a profundidad para llegar a una conclusión, misma que en un debate no es posible llegar.

Una vez establecido lo anterior, doy inicio a esta publicación.

Lenguaje incluyente
En la actualidad es muy mencionado el ser incluyente, el considerar a ambos géneros al expresarse, recuerdo que esto inició con los discursos del, entonces, presidente de México Vicente Fox Quesada, de quien era frecuente escuchar "chiquillos y chiquillas", y de ahí se fue extendiendo.

Ahora con las nuevas luchas de liberación femenina por querer ser tomadas en la misma consideración que los varones, y los movimientos, incluso sumamente agresivos y exhibicionistas que rayan lo absurdo, se desea resaltar aún más lo "excluyente" y sesgado que es el idioma, por lo que comienzan a inventarse toda clase de reglas para expresarse de forma incluyente.

Va "junto con pegado", porque ahora, y me parece que, a raíz de un video, se ha comenzado a querer meter inventos para disque incluir a quienes no se identifican con uno u otro género. Y por cierto, ahora se argumenta que ya no son "sexos", sino "géneros", y se menciona "el tercer sexo".

Entonces ¿El lenguaje es incluyente o no? Pues el Idioma español cuenta con más de 88,000 palabras más regionalismos, por lo que es una lengua muy rica.

El idioma castellano utiliza para referirse a grupos indefinidos en cuanto a su género el plural masculino, pero el hecho de ser el masculino basta para quienes argumentan la inequidad en el uso del idioma. Pero veamos ejemplos:

Singular	Plural
Masculino	Femenino	Neutro	Masculino	Femenino	Neutro
Bebé	Bebé	Bebé	Bebés	Bebés	Bebés
Infante	infanta	Infante	Infantes	infantas	Infantes
Niño	Niña	Niño	Niños	Niñas	Niños
Menor	Menor	Menor	Menores	Menores	Menores
Muchacho	Muchacha	Muchacho	Muchachos	Muchachas	Muchachos
Joven	Joven	Joven	Jovenes	Jovenes	Jovenes
Chico	Chica	Chico	Chicos	Chica	Chicos
Hombre	Mujer	Persona	Hombres	Mujeres	Personas
Individuo	Individuo	Individuo	Individuos	Individuos	Individuos
Caballero	Dama	Persona	Caballeros	Damas	Personas
 	Gente
Como se puede ver, existen suficientes términos completamente neutros que pueden ser utilizados de forma que se incluya a la totalidad del conjunto.

Singular	Plural
Masculino	Femenino	Neutro	Masculino	Femenino	Neutro
Un/Uno	Una	Un/Uno	Unos	Unas	Unos
Algún/Alguno	Alguna	Algún/Alguno	Algunos	Algunas	Algunos
Mucho	Mucha	Mucho	Muchos	Muchas	Muchos
Poco	Poca	Poco	Pocos	Pocas	Pocos
Todo	Toda	Todo	Todos	Todas	Todos
Estos son adverbios de cantidad, y sirven para indicar la misma de manera imprecisa sobre un conjunto especificado. La forma correcta es junto con la especificación del conjunto a que se refiere, sin embargo, dentro de un contexto donde ya se indicó dicho conjunto, suele utilizarse de forma independiente dejando sobre-entendido o de manera implícita hacia qué se refiere. Por ejemplo:
Todas las personas que asistieron a la fiesta estuvieron contentas, por lo que se entiende que todas (las personas) recibieron un buen trato.
Todos los individuos pertenecientes a la federación han demostrado su lealtad, por lo que a todos (los individuos) se les entregó un sencillo pero sincero reconocimiento.
En vez de decir (incorrectamente) "todas y todos" para enfatizar la inclusión de ambos géneros, y ahora "todes" para quienes se dicen no identificados con ningún género, lo correcto es "toda la gente", "todas las personas", "todos los individuos".
Las palabras que refieren a un rol o papel que desempeña una persona, tales como "estudiante", "comandante", "almirante", "gerente", "integrante", "navegante", "vigilante", "teniente", "adolescente", "constituyente", "presidente" son términos neutros, y la terminación "ante" o "ente" refiere a una entidad o individuo (persona) que realiza dicha labor.
Puesto que mi exposición no pretende ser un tratado lingüístico, sino una pequeña ayuda para un mejor uso de nuestra bella lengua. No profundizaré más allá. Estoy consciente que el idioma debe evolucionar, y lo hace al incorporar términos nuevos que tienen un significado perfectamente definido, términos que representan hitos particulares, tan intensos que merecen ser acuñados, por la referencia que conllevan, por ejemplo:
"cantinflear" De Cantinflas, popular actor mexicano y -ear. Hablar o actuar de forma disparatada e incongruente y sin decir nada con sustancia
El concepto de "tercer sexo"
Entre los comentarios que leí hacen mención al "tercer sexo", también argumentan el "todes" para referirse a personas que no se identifican con ser de sexo o género masculino o femenino.
En diversas sociedades se ha dado este concepto para referirse a personas cuyo género no está definido con claridad, ya sean de apariencia femenina con una mentalidad masculina o viceversa, o personas que se identifican como “no hombres” o “no mujeres”
Somos seres binomiales por decirlo de alguna forma, es decir, estamos formados por dos elemento o partes: una masculina y una femenina. No hay más, y todas las preferencias se encuentran dentro de la combinación de estas dos partes. fisiológicamente, físicamente, psicológicamente estamos formados por dos partes, aunque en diferente proporción.
Digamos que somos [m,f,r], donde m = masculino, f = femenino y r es su capacidad reproductiva, y manejamos una escala de 0 a 1 con 1 como totalmente y 0 nada. Una persona masculina capaz de reproducirse sería [1,0,1] y una femenina, también capaz de reproducirse sería [0,1,1].
El producto cruz de dos personas [m1, f1, r1] y [m2, f2, r2] que da origen a otro individuo [m3, f3, r3] es:
m3
f3
r3

m1
f1
r1
[f1*r2 – r1*f2]m3 + [m1*r2 – r1*m2]f3 + [m1*f2 - m2*f1]r3
m2
f2
r2


Acomodándolo, por economía de espacio, quedaría [m1 , f1 , r1] x [m2 , f2 , r2] = [m3 , f3 , r3]
Las columnas en rojo describen a nuevos individuos.
m
F
r
x
m
f
r
=
m
f
r
0
1
1

1
0
1

1.00
1.00
1.00
0.05
0.95
1

0.95
0.05
1

0.90
0.90
0.90
0.1
0.9
1

0.9
0.1
1

0.80
0.80
0.80
0.15
0.85
1

0.85
0.15
1

0.70
0.70
0.70
0.2
0.8
1

0.8
0.2
1

0.60
0.60
0.60
0.25
0.75
1

0.75
0.25
1

0.50
0.50
0.50
0.3
0.7
1

0.7
0.3
1

0.40
0.40
0.40
0.35
0.65
1

0.65
0.35
1

0.30
0.30
0.30
0.4
0.6
1

0.6
0.4
1

0.20
0.20
0.20
0.45
0.55
1

0.55
0.45
1

0.10
0.10
0.10
0.5
0.5
1

0.5
0.5
1

0.00
0.00
0.00
0.55
0.45
1

0.45
0.55
1

0.10
0.10
0.10
0.6
0.4
1

0.4
0.6
1

0.20
0.20
0.20
0.65
0.35
1

0.35
0.65
1

0.30
0.30
0.30
0.7
0.3
1

0.3
0.7
1

0.40
0.40
0.40
0.75
0.25
1

0.25
0.75
1

0.50
0.50
0.50
0.8
0.2
1

0.2
0.8
1

0.60
0.60
0.60
0.85
0.15
1

0.15
0.85
1

0.70
0.70
0.70
0.9
0.1
1

0.1
0.9
1

0.80
0.80
0.80
0.95
0.05
1

0.05
0.95
1

0.90
0.90
0.90
1
0
1

0
1
1

1.00
1.00
1.00

En todos los casos se da un individuo linealmente independiente de sus progenitores, sin embargo, se encuentra siempre dentro del mismo espacio que los define.
                Absolutamente todas las posibles combinaciones pueden ser llevadas hacia uno u otro extremo de la identidad. La existencia de un tercer sexo requiere la presencia de un sistema reproductor completamente independiente, que no pueda ser obtenido por la combinación de los otros dos.
                Análogamente: toda la gama de colores que nuestros ojos pueden percibir pueden ser formados por la composición de tres: rojo, verde y azul. Toda la gama de variantes humanas puede ser obtenida por la composición de dos. La existencia de este tercer sexo implica que para lograr la reproducción es necesaria la participación de estas tres partes, de lo contrario, una terminaría despareciendo.
                Es por ello que no es correcto el término "tercer sexo", así como tampoco es aplicable el término "todes", ya que, aunque como personas estamos formados por complejas estructuras, estas son femeninas o masculinas. Podemos modelas a una persona, de forma simple de la siguiente manera: 

Parte masculina
Parte femenina
Resultado
Cuerpo
A
B
e
Mente
C
D
f

Donde a los elementos {a, b, c, d} podemos darle cualquier calificación entre 0 y 1. Los resultados e y f, cualesquiera que sean, si quieren definirlos antes que los otros cuatro, pueden ser obtenidos como una combinación de los otros cuatro: e como la suma de a y b, mientras que f lo es respecto de c y d. Una persona puede ser completamente masculina, femenina, o una combinación de ambos.
Adicionalmente, la RAE define sexo (Del lat. sexus.): 1. m. Condición orgánica, masculina o femenina, de los animales y lasplantas. | 2. m. Conjunto de seres pertenecientes a un mismo sexo. Sexo masculino,femenino. | 3. m. Órganos sexuales.
Mientras que para Género da esta definición (Del lat. genus, -ĕris.) 1. m. Conjunto de seres que tienen uno o varios caracteres comunes. | 2. m. Clase o tipo a que pertenecen personas o cosas. Ese género de bromas no me gusta. | 3. m. Grupo al que pertenecen los seres humanos de cada sexo, entendido este desde un punto de vista sociocultural en lugar de exclusivamente biológico.
Por ello, en nuestro idioma, lo más adecuado es hablar de "tercer género" y no "tercer sexo".
Referencias
Real Academia Española, http://www.rae.es/
Tercer sexo, Wikipedia, https://es.wikipedia.org/wiki/Tercer_sexo

Producto vectorial, Wikipedia, https://es.wikipedia.org/wiki/Producto_vectorial
