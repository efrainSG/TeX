\section{Professional interviewing myself}
\subsection{Concepts}
\begin{multicols}{2}

\begin{enumerate}
%\item Si se requiere utilizar otro conjunto de separadores, ¿Qué cambios hay que realizar en el código?
%\item Si tengo que enfrentar un requerimiento con una tecnología nueva para mí, ¿Cuál es mi forma de proceder?
%\item Si en mi equipo, después de unos días de acoplamiento no me asignan tarea, ¿Cuál es mi forma de proceder?
%\item ¿Cuál ha sido mi experiencia con Protractor y Cucumber?
%\item ¿Por qué mi interés de aprenderlos?
%\item Preguntaron por algún error que, en mi experiencia, haya cometido y cómo lo afronté
%\item ¿Qué aprendí de ese error?
%\item Comentaron sobre un posterior uso de AWS y si tengo experiencia ahí.

\item What differences there are between the \emph{foreach} and \emph{for} loops? The first one consumes more resources than the second one. But depending of each situation, you can choice between them.
\item What is inheritance? it's the ability of derive classes taking another classes as base and inheriting the properties, methods and events from these prevoiusly defined classes or interfaces.
\item What is polymorphism? it's the ability to define different behaviors for a method according to its parameters. There are 2 types of polymorphism:
\begin{itemize}
\item Overloading is done using the same method name but with different signature.
\item Overriding is done by implementing a new behavior in the child class for a method defined in the parent class.
\end{itemize}
\item What is abstraction? Consists in making accesible to the user only the things that really matters. The user doesn't need to know how certain value is calculated or what others methods exist inside the class. In C\#, abstract classes can be defined by using the \emph{abstract} keyword and interfaces.
\item What is encapsulation? It's about hidding data or functions that the user isn't allowed to manipulate leaving only those methods defined as public and the data accesible only by properties. All those thigns must be grouped into a class.
\item What is an interface? It's a kind of contract that obligates a class to implement the methods and properties specified by it. It only contains properties and method signatures but not contains implentation. The use of interfaces allows multiple inheritance whereas by classes only allows single inheritance.
\item What is an abstract class? It's a particular type of class that contains abstract methods without implementation and regular methods with implementation. This class can't be instantiated and it must be inherited by another class and the child class needs to provide the implementation for the abstract methods.
\item Difference between interface and an abstract class? An interface doesn't have any implementation while an abstract class can have implementation of methods. One class can be inherited from only one abstract class but can inplements multiple interfaces.
\item What are the SOLID principles?
\begin{description}
\item[Single resposibility] Each class must be in charge of only one task, the same with the methods, each one must be responsible for only one job.
\item[Open-close]  The classes must be closed to modifications and at the same time opened to changes and new functions. One way could be declaring the clean and functional class as an abstract one, so, new classes can be created derived from the abstract class to implement the changes.
\item[Liskov substitution] In fact, this principle says: everything that can be done by a class, then its children classes can do regardless of the child class that it's been used. In other words, if a class is been used, then it's possible change it by an object instanciated from a derived class of the class originally used.
\item[Interface segregation] It's better create smaller well delimited interfaces that can be implemented in the classes rather than fewer bigger interfaces, this allows to implement only those methods that are required instead of implements all the methods, including the unnecessary.
\item[Dependency inversion] Refers to deacoplate the code between classes, it means that each class doesn't depends upon other classes to do its job and its recommended the use of constructors with interfaces instead of specific classes.
\end{description}
\item What is static? Declared variables or methods as static are globally accessible without creating an instance of the class.
\item What is the stack? Is a space inside the memory that stores methods, parameter, variables according the use of them and in a LIFO structure (Last-in, First-out). When a new method is invoked, its added to the stack and after that its variables are pushed in the top of the stack. When the method finishes, the stack pops-out them.
\item What is the heap? Is another space in the memory with te porpuse of allocate the objects and their references are stored in the stack.
\item What is a Jagged Array? It is an array of arrays where each element can have different length. This is opposite to multidmiensional arrays where each element is an array and every array has exactly the same amount of items.
\item What is serialization? Serialization is the process of converting an object into a form that can be persisted or transported like JSON, XML or binary.
\item What is deserialization? It is the process of converting an stream of data into an object.
\item What is a reference type? Reference type is a pointer to another memory location where the object is being stored.
\item What is a value type? A value type holds a data value within its own memory space.
\item What is boxing? It's about cast a specifict class object as a superclass Object.
\item What is unboxing? It's de opposite to boxing and refers to extract from an Object to its specific original class.
\item What are Generics? Generics allow to create classes, methods and properties using the type parameter without the specific data type. A type parameter is a placeholder for a particular type specified when creating an instance of the generic type. Generics increase the reusability of the code. You don't need to write code to handle different data types. Generics are type-safe. You get compile-time errors if you try to use a different data type than the one specified in the definition.
Generic has a performance advantage because it removes the possibilities of boxing and unboxing.
\item What are Delegates? They are references to methods, this allows to associate certain variable types to those methods, in this way they can be used in the responses of the methods. Multicast delegate? It is a delegate that points to multiple methods.
\item Which verbs do you know and when do you use those verbs?
\begin{description}
\item[GET] it's used in methods that retrieve information.
\item[POST] if the method do a data insertion, this verbs must be used.
\item[PUT] To update of almost all the data of the object.
\item[DELETE] is used to perform a deletion of the data and return a result.
\item[PATCH] updates some data of the object.
\end{description}
\item What is REST? Representational State Transfer, it means, the data transfer must be uniform, the API must avoid expose too much server details, client \& server must be separated, the request must be stateless.
\item What is SOAP? it means \emph{Simple Object Access Protocol} and refers to a XML based structure to represent, manage and transfer data objects across the internet.
\item What is CORS? It's an specification for information sharing between trustworthy domains. By default sharing information between domains is not allowed in terms of security but the confidence of the involved domains can be specified in the configuration to allow CORS.
\item How can you implement security for an API? Through the attribute Authorize and generating tokens that can be passed as parameters in the headers of each method call to de API, this token can be authenticated.
\item Can you explain media type formatters?They are built-in classses used to serialize informathion therefore, the APIs can understand the request data format and send it in the format as the client expects.
\item What are WebApi Filters? They are attributes that allows to implement certain added logic before or after API methods execution. Those filters can be applied to the API or to some methods. Every filter must implements the IFilter interface, basically.
\item How can you send parameters to an endpoint? It-s possible to pass data as parameters values to an endpoint by adding the \emph{FromBody} attribute to each one.
\item What is middleware? it's a software that allows the interaction between two or more different software systems.
\item Dependency injection: It's a design pattern based on provide certain objects to a customer class instead of creating them by the class itself.
\item What is a microservice? Software development method that structures an application as a collection of services that interact between them and where each microservice is focused on a specific domain, have its own database and they are loosely coupled. They are highly scalable and have one dedicated team, and the faults remains isolated. It's possible to develop each microservice in different technologies and languages, and they still can work together through APIs. In contrast, the communication can be difficult, also testing and debugging, and at the same time, develop microservices involves more resources.
\item What is a monolith application? It's a system designed as a whole big block that perform all the functionalities for the application. at the beginning, the application is easy to debug, modify, test and deploy, but along the time and after adding more and more capabilities, the risk of modify incorrectly increase, and the deployment must be done of the whole application.
\item docker deployment? It is an open source technlogy for creating containers. It allows to deploy applications using containers that contain the application and the required dependencies.
\item Pipeline are a set of chained processes that performs complex repetitive task, for example: deploying, testing, building, packaging. They're used in CI/CD approach.
%\item Containers
%\item Jenkins
%\item Encryption
\item Unit Tests
\begin{description}
\item[Fake] are objects that have working implementations, but not same as production one. Usually they take some shortcut and have simplified version of production code.
\item[Mock] are objects that register calls they receive.
In test assertion we can verify on Mocks that all expected actions were performed.
\item[Stub] is an object that holds predefined data and uses it to answer calls during tests. It is used when we cannot or don’t want to involve objects that would answer with real data or have undesirable side effects.
\item[Assert] it's a verification about the execution of a test and checks if the results are Ok on not.
\end{description}
\item front end
\item CI/CD: Means \emph{Continuous integration - Continuous deployment} and refers to cycles where the changes and updates done to a software are uploadad, tested, integrated and deployed in a continuous way through \emph{pieplines} that automate those parts of the process in order to gain agility and eficiency.
\end{enumerate}
\end{multicols}

\subsection{Architecture Patterns}
\begin{multicols}{2}
\begin{enumerate}
\item What is the Singleton pattern? It's a creational design pattern that is used to restrict to only one instance of a class available to all clients. It's possible to implement it by declaring a static singleton instance and making the constructor private and create a \emph{getInstance} method that returns the existing instance if it exists and if not it creates a new singleton instance.
\item What is the Factory pattern? It's a creational design pattern that defines an interface for creating objects and the client decides which class to instantiate. It hides the implementation logic for the creation of objects from the client.
\item What is the Facade pattern? It's a structural design pattern used to provide a simplified interface to a complex subsystem or set of subsystems.
\item What is the adapter pattern? This pattern consists in the creation of a class that is capable to contain a smaller class as a property, and its properties matches with other bigger class which it has to colaborate with.
\item backend for frontend (bff) pattern: It's an UI/UX oriented pattern design similar to API Gateway pattern. BFF aims to create an intermediate layer splitted in different APIs focused on each client type.
\item API Gateway pattern relies on creating an API that acts as proxy between client apps and the services, including microservices.
\end{enumerate}
\end{multicols}

\subsection{Amazon Web Services}
\begin{multicols}{2}
\begin{description}
\item[S3] This service provide buckets or storage spaces to storage resources such as images, videos and other type of files. Those files can be accessed, previous authorization, by other servies and applications.
\item[Load Balancer] Is a service that helps to balance the traffic along the active instances of an application keeping it quick and available.
\item[Autoscaling] This service is capable to create new instances of the application, even can increase the amount of resources, according the traffic received by the application. The same is in the opposite, if the application receive less traffic or requires less resources, this service can reduce the resources consumption and can stop unused instances.
\item[Cloudformation] Is a techology that can be used to specify through an script the configuration of new instances of the application. This ease the task of delpoy new servers or new instances of the application across different regions, for example.
\item[EC2] Is a web service thar provides resizable computing capacity. Provides the following features:
\begin{itemize}
\item Virtual computing environments, known as instances
\item Preconfigured templates for your instances, known as Amazon Machine Images (AMIs), that package the bits you need for your server (including the operating system and additional software)
\item Various configurations of CPU, memory, storage, and networking capacity for your instances, known as instance types
\item Secure login information for your instances using key pairs (AWS stores the public key, and you store the private key in a secure place)
\item Storage volumes for temporary data that's deleted when you stop, hibernate, or terminate your instance, known as instance store volumes
\item Persistent storage volumes for your data using Amazon Elastic Block Store (Amazon EBS), known as Amazon EBS volumes
\item Multiple physical locations for your resources, such as instances and Amazon EBS volumes, known as Regions and Availability Zones
\item A firewall that enables you to specify the protocols, ports, and source IP ranges that can reach your instances using security groups
\item Static IPv4 addresses for dynamic cloud computing, known as Elastic IP addresses
\item Metadata, known as tags, that you can create and assign to your Amazon EC2 resources
\item Virtual networks you can create that are logically isolated from the rest of the AWS Cloud, and that you can optionally connect to your own network, known as virtual private clouds (VPCs)
\end{itemize}
\item[SQS] Amazon Simple Queue Service (Amazon SQS) offers a secure, durable, and available hosted queue that lets you integrate and decouple distributed software systems and components. Amazon SQS offers common constructs such as dead-letter queues and cost allocation tags. It provides a generic web services API that you can access using any programming language that the AWS SDK supports. Amazon SQS supports both standard and FIFO queues.
\item[CodeBuild] AWS CodeBuild is a fully managed build service that compiles your source code, runs unit tests, and produces artifacts that are ready to deploy.
\item[ECS] Amazon Elastic Container Service (Amazon ECS) is a highly scalable, fast, container management service that makes it easy to run, stop, and manage Docker containers on a cluster of Amazon EC2 instances.  With Amazon ECS, your containers are defined in a task definition that you use to run an individual tasks or task within a service.
\item[SNS] Amazon Simple Notification Service (Amazon SNS) is a fully managed messaging service for both application-to-application (A2A) and application-to-person (A2P) communication. The A2A pub/sub functionality provides topics for high-throughput, push-based, many-to-many messaging between distributed systems, microservices, and event-driven serverless applications. Using Amazon SNS topics, your publisher systems can fanout messages to a large number of subscriber systems, including Amazon SQS queues, AWS Lambda functions, HTTPS endpoints, and Amazon Kinesis Data Firehose, for parallel processing. The A2P functionality enables you to send messages to users at scale via SMS, mobile push, and email.
\item[Cloudwatch] Amazon CloudWatch is a monitoring and observability service built for DevOps engineers, developers, site reliability engineers (SREs), IT managers, and product owners. CloudWatch provides you with data and actionable insights to monitor your applications, respond to system-wide performance changes, and optimize resource utilization. CloudWatch collects monitoring and operational data in the form of logs, metrics, and events. You get a unified view of operational health and gain complete visibility of your AWS resources, applications, and services running on AWS and on-premises. You can use CloudWatch to detect anomalous behavior in your environments, set alarms, visualize logs and metrics side by side, take automated actions, troubleshoot issues, and discover insights to keep your applications running smoothly.
\item[Lambda] AWS Lambda is a serverless, event-driven compute service that lets you run code for virtually any type of application or backend service without provisioning or managing servers. You can trigger Lambda from over 200 AWS services and software as a service (SaaS) applications, and only pay for what you use.
\item[cloudfront] 
\item[cdn] 
\item[SES]
\end{description}
\end{multicols}

\subsection{JavaScript}
\begin{multicols}{2}
\begin{enumerate}
\item difference between $==$ and $===$ operators: double equal performs a comparison between only values while triple equal operator compares values and data types.
\item inheritance $-->$ JS uses prototype.
\item variables scopes
\begin{itemize}
\item \emph{var} performs a declaration of a variable as global if it's used at the top of the code or inside the function.
\item \emph{let} allows to create a variable that lives only inside the block: function or loop body, for example.
\item \emph{const} declares a constant that stores a value which can't be changed, although if the value is an object, then its properties values can be updated.
\item \emph{local} refers a variable that lives inside a function.
\item \emph{block} refers a variable that is active only inside the block where is declared.
\item \emph{global} refers to a variable or constant declared in the top of the application.
\end{itemize}
\item functions types declarations
\begin{itemize}
\item Function expression: are declared using a variable format declaration, this allows tracing easily.
\item Anonymous functions: are used as parameters and/or values for variables. This kind of functions can't be recalled without an identifier.
\item Immediately invoked function expressions: are functions that are invoked in the same moment they're created, for example wrapping an anonymous function in parentheses and ending with a semicolon.
\item Constructor functions: allows to create function objects.
\item Arrow functions: Are similar to expression functions using this notation: \emph{let name} $=$ (arguments) $=>$ \{$/*$ code $*/$\}.
\end{itemize}
\item Explain the \emph{this} keyword: Refers to the object that the function is a property of. The value of ``this'' keyword will always depend on the object that is invoking the function.
\item Which data types do you know in Javascript? String, number, bigint, null, undefined, symbol.
\item What is the difference between undefined and null? an undefined value means that the variable doesn't exists while a null value means that the variable exists but without a value.
\item What is a closure? Closures is an ability of a function to remember the variables and functions that are declared in its outer scope.
\item promise: operation that will be completed.
\item Hoisting occurs when a variable, constant or function is used before its declaration, in this way, the compiler performs a prior declaration inside the upper scope.
\item New features in ES2015 or ES 6
\item asyncronous operations in Js: It's possible to perform asynchronous operations through the \emph{async, await} keywords and \emph{promises}. The async keyword enables the function to be executed in an asynchronous way, the await keyword allows to wait until the asynchronous operation finishes, and the promises are used to return and process the result of the asynchronous operations.
\item Falsy values: They are values that if are evaluated, then result in \emph{false}. There are six: $0$, \emph{Nan}, \emph{null}, \emph{undefined} and empty string.
\end{enumerate}
\end{multicols}

\subsection{React}
\begin{multicols}{2}
\begin{enumerate}
\item Components: Are elements that can be rendered in the UI. React has two kinds of components: \emph{class components} that can do a lot of operations, and \emph{functional components} that are lighter and do simpler jobs.
\item Lifecycles: are the life cycle for the components. most recent version includes these functions: \emph{componentDidMount, render, componentDidUnmount} and the previous version includes two more functions: \emph{getDefaultProps, getInitialState}. Between \emph{componentDidMount} and \emph{render} React executes \emph{componentWillReceiveProps, shouldComponentUpdate, componentWillUpdate}.
\item Hooks: Are special functions that allows React links statuses and lifecycle from function components. They don't  work inside class components.
\item JSX: Is an extension for JS that is used in React and allows write components that looks like HTML code inside javascript code.
\end{enumerate}
\end{multicols}

\subsection{Databases}
\begin{multicols}{2}
\begin{enumerate}
\item Functions VS Stored Procedures: Both can perform database operations, the difference are: functions must return a result, scalar or data sets and procecures optionally returns one or more results through output parameters.
\item Delete VS Truncate VS Drop: Delete erases records optionally specified in the \emph{where} clause, Truncate cleans the table and reset identity fields, Drop deletes data and structure of a table.
\item triggers: function that can be executed during or after an operation that is done in a table.
\end{enumerate}
\emph{Try to no mention a lot about previous projects. concise and short, say what you need and an example, if the answer is too long the interviewers can be lost in the answer}
\end{multicols}

\subsection{C\#}
\begin{multicols}{2}
\begin{enumerate}
\item What is .NET Core? It's a set of packages that allows to develop lighter, faster and more efficient, even cross platform, applications using the same technology than the .NET framework rather than loading the whole framework.
\item How do you consume an API from C\#? thorugh an HttpClient
\item What is the difference between ApiController and Controller? An APIController is a controller specialized to return data following a different routing scheme providing REST-ful by convention. Whereas, a Controller is capable to return a view as result.
\item How do you setup the dependency injection in .NET Core? Inside the \emph{Startup.cs} file must be declared the objects that should be used across the application and specify them as Singletons (one instance for all the requests regardless of the application) to avoid create additional and unnecessary instances, but according the necessities they could be Transient (one instance per request) or Scoped (one instance for all the request in the same application, changes between applications). Those objects must be passed to the related classes as parameters in the constructor methods.
\item Sealed: this keyword is used to specify that a class can't be inherited or a method can't be overrode. Stablishing a class ot method as sealed is a well practice if the class will be static, the same applies to the methods.
\item CLR compiles the code to the same intermediate language, regardless the programming language, prior to transform it to the machine language.
\item CLI provides a set of commands to perform tasks such as build, deploy or test.
\item MVC stands for Model View Controller. It is an architecture to build applications. It has 3 components of MVC: the model, the view, and the controller.
\begin{description}
\item[Model] They hold data and its related logic. It handles the object storage and retrieval from the databases for an application. 
\item[View] handles the UI part of an application. They get the information from the models for their display. For example, any employee view will include many components like text boxes, dropdowns, etc.
\item[Controller] They handle the user interactions, figure out the responses for the user input and also render the final output. For instance, the Employee controller will handle all the interactions and inputs from the Employee View and update the database using the Employee Model.
\item[partial view]
\end{description}
\item EF: is an ORM developed by Microsoft to interact easily with the database using the records as objects, the tables and views as classes and the stored procedures and functions as methods.
\item Code First vs Database First: ``Code first'' refers to create different classes to define the database structure by using \emph{migrations} and database updatings though different commands from the CLI; in the other hand, ``Database first'' is create the database directly and after generate the model upon the structure.
\item Add nuget package Entity Framework Core Design
\item scaffold-dbcontext "connectionstring" microsoft.entityframework.core.sqlsever -outputdir models extract classes from the database. 
\item Reflection 
\item Extensions Methods

\end{enumerate}
\end{multicols}

\subsection{Exercices}
\begin{multicols}{2}

%trade in wfe
%digital retailing
%nada
\begin{enumerate}
\item Write a solution that counts the words in a string and tells you how many times each word has occured. You can assume words are separated by a space, for example:
\begin{verbatim}
Input: My dog is a good dog
Output:
My: 1
dog: 2
is: 1
a: 1
good: 1
\end{verbatim}

\item Implement an algorithm to determine if a string has all unique characters, for example:
\begin{verbatim}
Input: ``abcdef''
Output: ``UNIQUE''
Input: ``abcdeedd''
Output: ``NOT UNIQUE''
\end{verbatim}

\item How would you ensure functions are called in the correct order if they're able to be called from different threads? For example:
\begin{verbatim}
public class Foo{
  public Foo(){}
  public First(){}
  public Second(){}
  public Third(){}
}
\end{verbatim}
How would you ensure First executes first, Second second, and so on.

\item Count different items in an array:
\begin{verbatim}
Input: chars = ["a","a","b","b","c","c","c"]
Output: 2a2b3c
\end{verbatim}

\item Reverse a string.
\end{enumerate}	
\end{multicols}
