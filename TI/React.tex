% !TEX TS-program = pdflatex
% !TEX encoding = UTF-8 Unicode

% This is a simple template for a LaTeX document using the "article" class.
% See "book", "report", "letter" for other types of document.

\documentclass[10pt]{article} % use larger type; default would be 10pt

\usepackage[utf8]{inputenc} % set input encoding (not needed with XeLaTeX)

%%% Examples of Article customizations
% These packages are optional, depending whether you want the features they provide.
% See the LaTeX Companion or other references for full information.

%%% PAGE DIMENSIONS
\usepackage{geometry} % to change the page dimensions
\geometry{a4paper} % or letterpaper (US) or a5paper or....
% \geometry{margin=2in} % for example, change the margins to 2 inches all round
% \geometry{landscape} % set up the page for landscape
%   read geometry.pdf for detailed page layout information

\usepackage{graphicx} % support the \includegraphics command and options

% \usepackage[parfill]{parskip} % Activate to begin paragraphs with an empty line rather than an indent

%%% PACKAGES
\usepackage{booktabs} % for much better looking tables
\usepackage{array} % for better arrays (eg matrices) in maths
\usepackage{paralist} % very flexible & customisable lists (eg. enumerate/itemize, etc.)
\usepackage{verbatim} % adds environment for commenting out blocks of text & for better verbatim
\usepackage{subfig} % make it possible to include more than one captioned figure/table in a single float
% These packages are all incorporated in the memoir class to one degree or another...

%%% HEADERS & FOOTERS
\usepackage{fancyhdr} % This should be set AFTER setting up the page geometry
\pagestyle{fancy} % options: empty , plain , fancy
\renewcommand{\headrulewidth}{0pt} % customise the layout...
\lhead{}\chead{}\rhead{}
\lfoot{}\cfoot{\thepage}\rfoot{}

%%% SECTION TITLE APPEARANCE
\usepackage{sectsty}
\allsectionsfont{\sffamily\mdseries\upshape} % (See the fntguide.pdf for font help)
% (This matches ConTeXt defaults)

%%% ToC (table of contents) APPEARANCE
\usepackage[nottoc,notlof,notlot]{tocbibind} % Put the bibliography in the ToC
\usepackage[titles,subfigure]{tocloft} % Alter the style of the Table of Contents
\renewcommand{\cftsecfont}{\rmfamily\mdseries\upshape}
\renewcommand{\cftsecpagefont}{\rmfamily\mdseries\upshape} % No bold!

%%% END Article customizations

%%% The "real" document content comes below...

\title{React, apuntes.}
\author{Efra}
%\date{} % Activate to display a given date or no date (if empty),
         % otherwise the current date is printed 

\begin{document}
\maketitle

\section{Notas}

\subsection{Components and reactive updates}
\begin{enumerate}
\item Virtual DOM nodes and JSX
\item Props and State
\begin{itemize}
\item (props) $=>$ \{\}
\item $[val, setVal]$ = useState(initialVal)
\item Immutable props. Mutable state
\end{itemize}
\item ReactDOM.render
\begin{itemize}
\item $<Component/>$
\item DOM node
\end{itemize}
\item React events (onClick, onSubmit, \dots)
\item Functions and class components
\end{enumerate}

\section{Ejemplo 1}
\begin{verbatim}
function Display(props) {
  return (
    <div>{props.message}</div>
  );
}

function Button(props) {
  const handleClick = () => props.onClickFunction(props.increment);
  return(
    <button onClick={handleClick}>
      +{props.increment}
    </button>
  );
}

function App(){
  const [counter, setCounter] = useState(42);
  const incrementCounter = (incrementValue) => setCounter(counter+incrementValue);
  return (
    <div>
      <Button onClickFunction={incrementCounter} increment={1}/>
      <Button onClickFunction={incrementCounter} increment={5}/>
      <Button onClickFunction={incrementCounter} increment={10}/>
      <Button onClickFunction={incrementCounter} increment={100}/>
      <Display message={counter} />
    </div>
  );
}

ReactDOM.render(
  <App />, 
  document.getElementById('mountNode'),
);
\end{verbatim}

\section{Ejemplo 2}
\begin{verbatim}
const render = () => {
  document.getElementById('mountNode').innerHTML = `
<div>
  hello, HTML
  <input />
  <pre>${(new Date).toLocaleTimeString()}</pre>
</div>
`;

ReactDOM.render(
  React.createElement(
    "div",
    null,
    "Hello, React",
    React.createElement('input', null),
    React.createElement('pre', null,
                        (new Date).toLocaleTimeString()),
    ),
  document.getElementById('mountNode2'),
  );
}

setInterval(render, 1000);
\end{verbatim}

\section{Ejemplo 3}
\begin{verbatim}
console.log(this);

const testerObj = {
  func1: function() {
    console.log('func1', this);
  },
  
  func2: () => {
    console.log('func2', this);
  },
};

testerObj.func1();
testerObj.func2();
\end{verbatim}
Consola
\begin{verbatim}
Console was cleared
VM812:8 {id: 'PLAYGROUND'}
VM812:11 func1 {func1: ƒ, func2: ƒ}
VM812:14 func2 {id: 'PLAYGROUND'}
\end{verbatim}
\section{Ejemplo 4}

\section{Ejemplo 5}

\section{Ejemplo 6}

\end{document}
