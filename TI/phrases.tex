\section{Excercises about sentences}
\subsection{Present perfect simple tense}
\begin{multicols}{2}
Sentences that refers to actions that started in the past but continues during the present.\\
\emph{Noun + have(n't)/has(n't) + verb in participle}.\\

\begin{itemize}
\item \textcolor{blue}{I have painted} that painting. / \textcolor{blue}{I've painted} that painting.
\item \textcolor{blue}{I have\textbf{n't} planted} my seeds in the garden. / \textcolor{blue}{I've \textbf{not} planted} my seeds in the garden.
\item \textcolor{blue}{I have \emph{just} arrived} to home today. / \textcolor{blue}{I've \emph{just} arrived} to home today.

\item \textcolor{blue}{You have cleaned} the house this morning. / \textcolor{blue}{You've cleaned} the house this morning.
\item \textcolor{blue}{You have\textbf{n't} washed} your paint brushes here. / \textcolor{blue}{You've \textbf{not} washed} your paint brushes here.
\item \textcolor{blue}{You have \emph{just} finished} that painting in the wall. / \textcolor{blue}{You've \emph{just} finished} that painting in the wall.

\item \textcolor{blue}{He has climbed} those mountains. / \textcolor{blue}{He's climbed} those mountains.
\item \textcolor{blue}{He has\textbf{n't} read} the book he borrowed. / \textcolor{blue}{He's \textbf{not} read} the book he borrowed.
\item \textcolor{blue}{He has \emph{just} studied} a new math course. / \textcolor{blue}{He's \emph{just} studied} a new math course.

\item \textcolor{blue}{She has watched} that series recently. / \textcolor{blue}{She's watched} that series recently.
\item \textcolor{blue}{She has\textbf{n't} thrown} the coin in the fountain this time. / \textcolor{blue}{She's \textbf{not} thrown} the coin in the fountain this time.
\item \textcolor{blue}{She has \emph{just} crossed} the bridge in the park. / \textcolor{blue}{She's \emph{just} crossed} the bridge in the park.

\item \textcolor{blue}{We have finished} our project last night. / \textcolor{blue}{We've finished} our project last night.
\item \textcolor{blue}{We have\textbf{n't} done} the homework yet. / \textcolor{blue}{We've \textbf{not} done} the homework yet.
\item \textcolor{blue}{We have \emph{just} taken} the cab to the airport. / \textcolor{blue}{We've \emph{just} taken} the cab to the airport.

\item \textcolor{blue}{They have hiked} many times these paths. / \textcolor{blue}{They've hiked} many times these paths.
\item \textcolor{blue}{They have\textbf{n't} trained} enough for the competition. / \textcolor{blue}{They've \textbf{not} trained} enough for the competition.
\item \textcolor{blue}{They have \emph{just} married}. / \textcolor{blue}{They've \emph{just} married}.

\end{itemize}
\end{multicols}

\subsection{Present perfect continuous tense}
\begin{multicols}{2}
Sentences that emphasizes actions that continues at the moment one is speaking.\\
\emph{Noun + have(n't)/has(n't) + been + verb-ing }.\\

\begin{itemize}
\item \textcolor{blue}{I have been drawing} a lot of scketches in my notebook. / \textcolor{blue}{I've been drawing} a lot of scketches in my notebook.
\item \textcolor{blue}{I have been working} for this company since past May. / \textcolor{blue}{I've been working} here for almost one year.
\item \textcolor{blue}{I haven't been swimming} in the pool this week. / \textcolor{blue}{I've not been swimming} in the pool this week.
\item \textcolor{blue}{I have not been painting} since December. / \textcolor{blue}{I've not been swimming} since 2010.

\item \textcolor{blue}{You have been cooking} hotcakes for breakfast. / \textcolor{blue}{You've been cooking} hotcakes for breakfast.
\item \textcolor{blue}{You haven't been making} handycrafts for a while. / \textcolor{blue}{You've not been making} handycrafts for a while.
\item \textcolor{blue}{Have you been reading} all those books?

\item \textcolor{blue}{He has been writing} his notes carefuly. / \textcolor{blue}{He's been writing} his notes carefuly.
\item \textcolor{blue}{He hasn't been running} recently. / \textcolor{blue}{He's not been running} recently.
\item \textcolor{blue}{Has he been studying} for his final exam?

\item \textcolor{blue}{She has been painting} a very beautyful mural on the wall. / \textcolor{blue}{She's been painting} a very beautyful mural on the wall.
\item \textcolor{blue}{She hasn't been writting} in her daily since last month. / \textcolor{blue}{She's not been writting} in her daily since last month.
\item \textcolor{blue}{Has she been teaching} this course?

\item \textcolor{blue}{We have been riding} our bikes yesterday afternoon. / \textcolor{blue}{We've been riding} our bikes yesterday afternoon.
\item \textcolor{blue}{We haven't been running} this week. / \textcolor{blue}{We've not been running} this week.
\item \textcolor{blue}{Have we been hiking} at least two hours?

\item \textcolor{blue}{They have been training} since last season. / \textcolor{blue}{Tehy've been training} since last season.
\item \textcolor{blue}{They haven't been playing} soccer. / \textcolor{blue}{They've not been playing} soccer.
\item \textcolor{blue}{Have they been working} in this project?
\end{itemize}
\end{multicols}

\subsection{was/were + present participle sentences}
\begin{multicols}{2}

The present participle is formed by adding the ending ``–ing'' to the infinitive (dropping any silent ``e'' at the end of the infinitive):

\begin{itemize}
\item The present participle may often function as an adjective: That's an interesting book.
\item The present participle can be used as a noun denoting an activity (this form is also called a gerund): Swimming is good exercise.
\item The present participle can indicate an action that is taking place, although it cannot stand by itself as a verb. In these cases it generally modifies a noun (or pronoun), an adverb, or a past participle: Washing clothes is not my idea of a job.
\item The present participle is used in progressive verb tenses, which indicate continuing actions or actions in progress (the present progressive, the future progressive, the present perfect progressive): I am eating my dinner. He was walking across the park. We will be calling you tomorrow.
\item The present participle may be used with ``while'' or ``by'' to express an idea of simultaneity (``while'') or causality (``by''): He finished dinner while watching television. By using a dictionary he could find all the words. While speaking on the phone, she doodled. By calling the police you saved my life!
\item The present participle of the auxiliary ``have'' may be used with the past participle to describe a past condition resulting in another action: Having spent all his money, he returned home. Having told herself that she would be too late, she accelerated.
\end{itemize}
\begin{itemize}
\item I \textcolor{blue}{was studying} when you arrived.
\item I \textcolor{blue}{wasn't sleeping}. It's true!
\item \textcolor{blue}{Was} I \textcolor{blue}{reading} this old book?
\item You \textcolor{blue}{were drawing} a beautiful landscape, please finish it.
\item You \textcolor{blue}{weren't walking} by the mall.
\item \textcolor{blue}{Were} you \textcolor{blue}{traveling} when this happened?
\item He \textcolor{blue}{was dinning} when they surprised him with a cake.
\item He \textcolor{blue}{wasn't training} because he injured his back.
\item \textcolor{blue}{Was} he \textcolor{blue}{writting} his essay?
\item She \textcolor{blue}{was buying} decorations for the party.
\item She \textcolor{blue}{wasn't washing} the dishes.
\item \textcolor{blue}{Was} she \textcolor{blue}{doing} exercise this afternoon?
\item We \textcolor{blue}{were playing} an interesting football game when started raining.
\item We \textcolor{blue}{weren't walking} beacause it was raining.
\item \textcolor{blue}{Were} we \textcolor{blue}{talking} about web services the last week?
\item They \textcolor{blue}{were watching} TV while dinning.
\item They \textcolor{blue}{weren't listening} radio before the news broadcast.
\item \textcolor{blue}{Were} they \textcolor{blue}{arriving} while you was there?
\end{itemize}
\end{multicols}

\subsection{Past perfect tense}
\begin{multicols}{2}
\emph{Noun + \textbf{had} + verb in participle}
\begin{itemize}
\item I \textcolor{blue}{had been} in the park many times before. / I\textcolor{blue}{'d been} in the park many times before.
\item I \textcolor{blue}{hadn't swom} in the lake today because it's cold. / I\textcolor{blue}{'d not swom} in the lake today because it's cold.
\item \textcolor{blue}{Had} I \textcolor{blue}{seen} him today? Tell the truth.
\item You \textcolor{blue}{had worked} enough this week. / You\textcolor{blue}{'d worked} enough this week.
\item You \textcolor{blue}{hadn't bought} the sodas for the party. / You\textcolor{blue}{'d not bought} the sodas for the party.
\item \textcolor{blue}{Had} you \textcolor{blue}{written} your last chapter?
\item He \textcolor{blue}{had developed} his app this month. / He\textcolor{blue}{'d developed} his app this month.
\item He \textcolor{blue}{hadn't planed} his trip because he wants an adventure. / He\textcolor{blue}{'d not planed} his trip because he wants an adventure.
\item \textcolor{blue}{Had} he \textcolor{blue}{hiked} yesterday?
\item She \textcolor{blue}{had drawn} a beutiful landscape. / She\textcolor{blue}{'d drawn} a beautiful landscape.
\item She \textcolor{blue}{hadn't done} her duties. / She\textcolor{blue}{'d not done} her duties.
\item \textcolor{blue}{Had} she \textcolor{blue}{built} the next release?
\item We \textcolor{blue}{had brought} our things this morning. / We\textcolor{blue}{'d brought} our things this morning.
\item We \textcolor{blue}{hadn't fixed} the sofa yet. / We\textcolor{blue}{'d not fixed} the sofa yet.
\item \textcolor{blue}{Had} we \textcolor{blue}{studied} enough for the exam?
\item They \textcolor{blue}{had been} visiting their relatives. / They\textcolor{blue}{'d been} visiting their relatives.
\item They \textcolor{blue}{hadn't practiced} for the festival. / They\textcolor{blue}{'d not practiced} for the festival.
\item \textcolor{blue}{Had} they \textcolor{blue}{noticed} the changes?
\end{itemize}
\end{multicols}

\subsection{used to\dots sentences}
\begin{multicols}{2}
Refers to actions that occured in the past but not more in the present.\\
\emph{Noun + \textbf{used to} + verb in infinitive}\\
\emph{Noun + \textbf{didn't use to} + verb in infinitive}\\

\begin{itemize}
\item \textcolor{blue}{I used to play} piano when I was child.
\item \textcolor{blue}{I didn't use to swim} because I was too shy.
\item \textcolor{blue}{Did I use to research} when you was student?

\item \textcolor{blue}{You used to run} marathon few years ago.
\item \textcolor{blue}{You didn't use to hike} a lot like now.
\item \textcolor{blue}{Did you use to dance} in the school?

\item \textcolor{blue}{He used to do} software very complex.
\item \textcolor{blue}{He didn't use to draw} technical drawings.
\item \textcolor{blue}{Did he use to write} articles for educational magazines?

\item \textcolor{blue}{She used to sing} a lot of jazz songs.
\item \textcolor{blue}{She didn't use to drink} coffee too late in the night.
\item \textcolor{blue}{Did she use to make} handycrafts for her kids?

\item \textcolor{blue}{We used to play} videogames during vacation when we were children.
\item \textcolor{blue}{We didn't use to work} on vacation.
\item \textcolor{blue}{Did we use to travel} every vacation?

\item \textcolor{blue}{They used to climb} mountains.
\item \textcolor{blue}{They didn't use to train} on wednesday.
\item \textcolor{blue}{Did they use to buy} sweets often?
\end{itemize}
\end{multicols}

\subsection{Future tense}
\begin{multicols}{2}
\begin{itemize}
\item \textcolor{blue}{I will} learn React and AWS. / \textcolor{blue}{I'll} learn React and AWS.
\item \textcolor{blue}{I will not} run too much. / \textcolor{blue}{I won't} run too much.
\item \textcolor{blue}{Will I} go fishing?
\item \textcolor{blue}{You will} work hard. / \textcolor{blue}{You'll} work hard.
\item \textcolor{blue}{You will not} travel this month. / \textcolor{blue}{You won't} travel this month.
\item \textcolor{blue}{Will you} cooking the dinner?
\item \textcolor{blue}{He will} break the record. / \textcolor{blue}{He'll} break the record.
\item \textcolor{blue}{He will not} paint the house. / \textcolor{blue}{He won't} paint the house.
\item \textcolor{blue}{Will he}
\item \textcolor{blue}{She will}
\item \textcolor{blue}{She will not} play the piano. / \textcolor{blue}{She wont} play the piano.
\item \textcolor{blue}{Will she}
\item \textcolor{blue}{We will} play basketball tomorrow. / \textcolor{blue}{We'll} play basketball tomorrow.
\item \textcolor{blue}{We will not} dance the next week. / \textcolor{blue}{We won't} dance the next week.
\item \textcolor{blue}{Will we}
\item \textcolor{blue}{They will}
\item \textcolor{blue}{The will not}
\item \textcolor{blue}{Will they}
\end{itemize}
\end{multicols}

\subsection{``If'' clauses: Zero Conditional}
\begin{multicols}{2}
Refers to real facts.\\
\emph{\textbf{If} + simple present tense + `,' + simple present tense}
\begin{itemize}
\item \textcolor{red}{If} I hit the glass, it breaks.
\item \textcolor{red}{If} I don't practice, I fail the test.
\item \textcolor{red}{If} you push the red button, the parachute opens.
\item \textcolor{red}{If} you don't hear, you can't help.
\item \textcolor{red}{If} he swim everyday, he's happy.
\item \textcolor{red}{If} he doesn't stop doing his homework, he finishes it soon.
\item \textcolor{red}{If} she stop singing, her baby cries.
\item \textcolor{red}{If} she doesn't say the truth, I punish her.
\item \textcolor{red}{If} we study, we pass the exam.
\item \textcolor{red}{If} we don't wake up early, the bus leave us.
\item \textcolor{red}{If} they speak loud, we can hear them.
\item \textcolor{red}{If} they don't do their job, we can't take vacation.
\end{itemize}
\end{multicols}

\subsection{``If'' clauses: First Conditional}
\begin{multicols}{2}
Refers to situations that occurs in the present but the result will be, probably, in the future.\\
\emph{\textbf{If} + simple present tense + `,' + simple future tense}
\begin{itemize}
\item \textcolor{red}{If} I don't do excercise today, \textcolor{blue}{I'll} feel tired.
\item \textcolor{red}{If} I start my project today, \textcolor{blue}{I'll} finish it soon.
\item \textcolor{red}{If} you don't write your homework, \textcolor{blue}{you'll} forget it.
\item \textcolor{red}{If} you study frequently, \textcolor{blue}{you'll} have more spare time.
\item \textcolor{red}{If} he help us with our project, \textcolor{blue}{we'll} go to the park.
\item \textcolor{red}{If} he doesn't train enough, \textcolor{blue}{he'll} be last in the trial.
\item \textcolor{red}{If} she
\item \textcolor{red}{If} she
\item \textcolor{red}{If} we
\item \textcolor{red}{If} we
\item \textcolor{red}{If} they
\item \textcolor{red}{If} they
\end{itemize}
\end{multicols}

\subsection{Polite sentences}
\begin{multicols}{2}
\begin{itemize}
\item 
\item 
\item 
\item 
\item 
\item 
\item 
\item 
\item 
\item 
\end{itemize}
\end{multicols}

\subsection{Might, must sentences}
\begin{multicols}{2}
\begin{itemize}
\item I might
\item I must
\item Might I
\item Must I
\item 
\item 
\item 
\item 
\item 
\item 
\item 
\item 
\item 
\item 
\item 
\item 
\item 
\item 
\item 
\item 
\item 
\item 
\item 
\item 
\end{itemize}
\end{multicols}

\subsection{Should, Need, Must sentences}
\begin{multicols}{2}
\begin{itemize}
\item I should
\item I shouldn't
\item Should I
\item I need
\item I don't need
\item Do I need
\item I must
\item I must not
\item Must I
\item 
\item 
\item 
\item 
\item 
\item 
\item 
\item 
\item 
\item 
\item 
\item 
\item 
\item 
\item 
\item 
\item 
\item 
\item 
\item 
\item 
\item 
\item 
\item 
\item 
\item 
\item 
\item 
\item 
\item 
\item 
\item 
\item 
\item 
\item 
\item 
\item 
\item 
\item 
\item 
\item 
\item 
\item 
\item 
\item 
\end{itemize}
\end{multicols}

\subsection{Questions with frequency adverbs}
\begin{multicols}{2}
Adverbs of frequency are
\begin{description}
\item[0 \%] = Never.
\item[5 \%] = Almost never, rarely.
\item[20 \%] = Not very often, seldom.
\item[50 \%] = Sometimes, occasionally.
\item[80 \%] = Usually, often, frequently.
\item[95 \%] = Almost always.
\item[100 \%] = Always.
\end{description}
\begin{itemize}
\item 
\item 
\item 
\item 
\item 
\item 
\item 
\item 
\item 
\item 
\end{itemize}
\end{multicols}

\subsection{Expressions}
\begin{multicols}{2}
\begin{itemize}
\item That's easy enough / Too much
\item Could you help me? / Can / Will / Could / Would
\item Such a good advice / So / Such
\item A great mood
\item It's been a while / awhile / since / been awhile since...
\item Why don we...? / How about... / Let's...
\end{itemize}
\end{multicols}

\subsection{Quick questions}
\begin{multicols}{2}
\begin{itemize}
\item I'm a good artist, aren't I?
\item I wrote two books, didn't I?
\item I've seen that fountain before, haven't I?
\item I should do my homework, shouldn't I?
\item I shouldn't take other muffin, should I?
\item You're the winner, aren't you?
\item You washed the dishes yestarday, didn't you?
\item You've wrote this note, haven't you?
\item You should write another one, shouldn't you?
\item He is a professor, isn't he?
\item He teached you before, didn't he?
\item He've been professor for long time, hasn't he?
\item He should teach us, shouldn't he?
\item 
\item 
\item 
\item 
\item 
\item 
\item 
\item 
\end{itemize}
I should have visited\dots \\
I could have seen\dots \\
I might have enjoyed\dots \\
She would have loved\dots \\
If I went\dots I would want\dots \\
If I had\dots I would\dots \\
If I hadn't\dots I would(n't)\dots \\

\end{multicols}

\textcolor{lightgray}{They was very excited because they were showing an exciting movie.}\\
\textcolor{lightgray}{They was very excited because they were watching an exciting movie.}\\
\textcolor{lightgray}{They was very excited because they were playing an exciting movie.}