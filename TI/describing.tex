\section{Describing myself}

\subsection{Who am I?}
\begin{multicols}{2}
My name is Efrain and I like programming. I'm a software developer and I've worked on different administrative apps, moreover, I've worked for different companies in different places therefore I have knowledge of different languages and tools thanks to I had roles as back-end and front-end. Therefore, most of my knowledge was aquired through live experience instead of courses.\\
I hope someday create intelligent apps because my dream related to programming is to create technology that ease software development or develop some videogames.\\
I have experience as professor at a private university. I teached about networking, programming and databases, besides, I'd also have liked to teach methodology of researching or maths.\\
I love maths and I have interest in different physics concepts. I'd like to create a few robots.\\
Currently I have a lot of things to do inside my head and I'd like I could realize them.
\end{multicols}

\subsection{What are my capabilities? (nor related to work neither hobbies)}
\begin{multicols}{2}
My experience as professor allows me to teach others effectively, moreover, I can write small articles and essays, also, because of that experience, I'm capable to speak out to audiences, of course, having the context and waht to say. I can make handicrafts. I can make artistic and technical drawings. I can learn quickly.
\end{multicols}

\subsection{About time sparing}
\begin{multicols}{2}
I can't do many sports because a back injury, therefore I can do only swimming furthermore I love it, I love the feeling of freedom when I move in the water, I love how far I can go without pain.\\
I'd like practice karate to help my family and maybe teach my son and daugther, although I can't do most of the techniques, regardless of that, I'd try to develop a defense that help them to cope with bulliers.\\
Since my master degreee course, I like wrinting, consequently I put all my writings into one file and I created a small book. Now, my goal is to edit all my writings and create better articles and essays.\\
I like reading books. I've been reading books, articles and novels since I was very young. Sometimes I can feel how the story captures me too deep that I can't feel time passing. In other readings, the author makes me travel through the universe from and back to my seat and my mind blows up. Have you watched the first movie of Matrix? Do you remember how Neo feels when he recieves the first dose of digital knowledge? Sometimes I have a similar feeling and is delightful.\\
I like painting, but has been a while since my last painting.
I used to paint landscapes and I feel I'm not good enought to draw and paint human bodies. I was studying to paint when I was youngh and I had more leisure. I know how to paint using pencil colors and oil paint but I know that I need to improve that skill perhaps, in few time I'll restart my old hobby because I loved it and I miss to do that.\\
On the other hand\dots
I like travel.
I like go out to different, and maybe unknown, places.
I enjiy viewing different things and possibly buy few small stuffs, not very expensive but beautiful things, also, I like to taste different food and to take some photos.
Actually I don't feel I could do that without creatin a new debt and that makes me feel sad. Really I'd like to travel again and feel not worries about the money, to feel a comfortable place to sleep and wake up early to go breakfast, try a new place to eat and where to go dinner. I'd like to travel and visit new places and return to home completly relaxed.\\
Also, I like to be well dressed. I used to wear formal shirt everyday because of my previous jobs and I liked. I reached to have thirty different shirts with a lot of different ties, four suits, two pair of shoes, more than ten polo style t-shirts.
I really miss these manner to dress. I, actually, have just one suit, 7 shirts and one pair of shoes. Of course, talking about formal clothes.\\


\end{multicols}

\subsection{My old habits}

\subsection{What projects did I do?}
\begin{multicols}{2}
\begin{description}
\item[Lematizer]
\item[Di-JAVA]
\item[Dextra] This was a management sysem for universities. It was developed using a three layer architecture with SQL Server 2008 as DBMS, Linq as  its ORM and WPF for the user interfaces. The rest of the application was developed as a Class Library. Dextra used .NET Framework 3.5 and 4. Also, it had a small part designed using the Factory pattern. Additional, the system had a small web platform where teachers could register evaluation scores for their students, their plannings and rules for each course, also they could update their profiles, consult their schedules, administrative personel could look up for evaluations, plannings, course rules and do another tasks, and studens could review their evaluations, schedules, plannings and rules for each course they was taking. That system took me two years to be completed because the whole project was developed only by myself.
\item[Library system]
\item[Cotediv] This was designed to be a ludic educational project combining a few characteristics based on FaceBook, Twitter, and mind maps systems. For that last element it required a language process to link the information. The system was developed using PHP, JavaScript and mySQL, but none frameworks was used. The system allows to the participants to take questionnaires and test to get points, also each concept published (the main purpose) could be valuated and evaluated by other participants and professors allowing to the participants to get more points and follow up their progress as ranking tables and automated mind maps.
\item[Clinic stories]
\end{description}
\end{multicols}
\subsection{What publishings did I do?}

\subsection{What are my current (at the moment to doing this writings) goals?}

[\section{Professional interviewing myself}]
\begin{multicols}{2}
\begin{enumerate}
\item ¿Qué diferencia hay entre foreach y for en C\#? \emph{"foreach" consume más recursos que "for", pero dependiendo de las situaciones se utiliza uno u otro. En mi caso fue para reducir cantidad de comparaciones.}
\item Si se requiere utilizar otro conjunto de separadores, ¿Qué cambios hay que realizar en el código?
\item Si tengo que enfrentar un requerimiento con una tecnología nueva para mí, ¿Cuál es mi forma de proceder?
\item Si en mi equipo, después de unos días de acoplamiento no me asignan tarea, ¿Cuál es mi forma de proceder?
\item ¿Cuál ha sido mi experiencia con Protractor y Cucumber?
\item ¿Por qué mi interés de aprenderlos?
\item Preguntaron por algún error que, en mi experiencia, haya cometido y cómo lo afronté
\item ¿Qué aprendí de ese error?
\item Comentaron sobre un posterior uso de AWS y si tengo experiencia ahí.
\item What is inheritance? it's the ability of create classes taking another classes as base, in other words, it allows to derive new classes from other prevoiusly defined classes or interfaces.
\item What is polymorphism? it's the ability to define different behaviors for a method according to its parameters. 
\item What is abstraction? don't need know how certain class or method works to use it. The user of our classes or methods doesn't need to know how they work to use them.
\item What is encapsulation? it's hide every data or function that the user must not manipulate leaving accesible just those properties or methods allowed for that purpose.
\item What is an interface? -OK-
\item What is an abstract class? -OK-
\item Difference between interface and an abstract class? -OK-
\item Singleton -OK-
\item Factory -OK-
\item SOLID: Single, Open-close, Liskov substitution, Interface segregation, Dependency inversion. -OK-
\item What is static? Static declared variables or methods are globally accessible without creating an instance of the class
\item What is serialization? Serialization is the process of converting an object into a form that can be persisted or transported like JSON, XML or binary
\item What is deserialization? It is the process of converting an stream of data into an object
\item What is a reference type? Reference type is a pointer to another memory location where the Object is being stored
\item What is a value type? A value type holds a data value within its own memory space
\item What is boxing? cuando encapsulo un elemento de un tipo específico en un Object.
\item What is unboxing? Cuando extraigo de un elemento Object hacia el tipo específico a fin de procesarlo.
\item What are Generics? 
\item Delegates? Son referencias hacia métodos, esto permite asociar a cierto tipo de variables esos métodos, de esta manera se utilizan en las respuestas de eventos.
\item Which verbs do you know and when do you use those verbs? GET, POST, PUT, DELETE
\item How do you implement PATCH? Used to update/modify elements, if the element exists with the same values then no operation must be performed, else, update the info.
\item What is REST? Representational State Transfer, it means, the data transfer must be uniform, the API must avoid expose too much server details, client \& server must be separated, the request must be stateless.
\item What is SOAP?
\item Can you explain media type formatters?
\item What are WebApi Filters?
\item What is CORS?
\item How do you consume an API from C\#?
\item How can you implement security for an API?
\item What is the difference between ApiController and Controller?
\item What is a microservice?
\item What is a monolith application?
\item aws
\item lambda
\item difference .net and .net core
\item docker deployment
\item pipeline
\item containers
\item jenkins
\item unit tests
\item front end
\item Explain the this keyword: The “this” keyword refers to the object that the function is a property of.The value of “this” keyword will always depend on the object that is invoking the function.
\item Which data types do you know in Javascript: String, number, bigint, null, undefined, symbol
\item What is the difference between undefined and null
\item What is an Immediately Invoked Function in JavaScript? An Immediately Invoked Function ( known as IIFE and pronounced as IIFY) is a function that runs as soon as it is defined.
\item What is a closure Closures is an ability of a function to remember the variables and functions that are declared in its outer scope.
\item New features in ES2015 or ES 6
\end{enumerate}
\end{multicols}
