\section{Describing myself}

\subsection{Who am I?}
\begin{multicols}{2}
Let me introduce myself. Well, it's a little difficult to me talk about myself.

I'm someone that enjoys learning new things, very different things, maybe that's the reason why I believe that describing me it's complicated. I love reading and writing. I put all my writings together and I discovered that they could form a small book, regardless from my two theses. I have two blogs, one is about IT and the other refers to different topics including costumes, traditions reflections and tech topics.

Moreover, I like mathematics a lot; sometimes you can find me solving Calculus exercises, writing new content for my anthology (I combine my two loves: writing and maths). Sometimes my wife told me that maths is my escape from reality, however, even when maths could be an escape for me, it allows me to re-focus my thinking and to find certain balance.

I like drawing and painting, few years ago I made some oil paitings, recently I'm taking up this hobby again; I'm thinking in create a painting with different tiles but I don't have the full idea, just a scketch.

My new hobby is to bind my own notebooks, books, albums. It's fun and interesting, I can reuse and recycle used paper sheets and the result is nicelooking, interesting and curious. Briefly I started to do some pop-ups handcrafts that could be bound; mixed with some of my writings could result into an interesting creation.

I'm a software developer and I've been database administrator in few projects, it really like me create software, it's something strange seeing how from nothing the systems comes alive. Also I'm an amateur painter, I made some paintings but none were sold. I paint with oleum preferently landscapes by imaginating them. I worked as teacher few years, actually I can say that I prefer to teach you, husband, father, brother, son, friend\dots
\end{multicols}

\subsection{What are my capabilities? (not related to work and no hobbies}
\begin{multicols}{2}
I'm capable to: teach others, make drawings and paintings. I have hability to make paper handicrafts. I'm capable to write books, articles, essaies.
\end{multicols}

\subsection{About time sparing}
\begin{multicols}{2}
swimming, drawing, painting, binding, doing excercises, looking for new clothes, reading, writing, pop-up handicrafts.
\end{multicols}

\subsection{What projects did I do?}
\begin{multicols}{2}
\begin{description}
\item[Lematizer]
\item[Di-JAVA]
\item[Dextra] This was a management sysem for universities. It was developed using a three layer architecture with SQL Server 2008 as DBMS, Linq as  its ORM and WPF for the user interfaces. The rest of the application was developed as a Class Library. Dextra used .NET Framework 3.5 and 4. Also, it had a small part designed using the Factory pattern. Additional, the system had a small web platform where teachers could register evaluation scores for their students, their plannings and rules for each course, also they could update their profiles, consult their schedules, administrative personel could look up for evaluations, plannings, course rules and do another tasks, and studens could review their evaluations, schedules, plannings and rules for each course they was taking. That system took me two years to be completed because the whole project was developed only by myself.
\item[Library system]
\item[Cotediv] This was designed to be a ludic educational project combining a few characteristics based on FaceBook, Twitter, and mind maps systems. For that last element it required a language process to link the information. The system was developed using PHP, JavaScript and mySQL, but none frameworks was used. The system allows to the participants to take questionnaires and test to get points, also each concept published (the main purpose) could be valuated and evaluated by other participants and professors allowing to the participants to get more points and follow up their progress as ranking tables and automated mind maps.
\item[Clinic stories]
\end{description}
\end{multicols}
\subsection{What publishings did I do?}

\subsection{What are my current (at the moment to doing this writings) goals?}

[\section{Professional interviewing myself}]
\begin{multicols}{2}
\begin{enumerate}
\item ¿Qué diferencia hay entre foreach y for en C\#? \emph{"foreach" consume más recursos que "for", pero dependiendo de las situaciones se utiliza uno u otro. En mi caso fue para reducir cantidad de comparaciones.}
\item Si se requiere utilizar otro conjunto de separadores, ¿Qué cambios hay que realizar en el código?
\item Si tengo que enfrentar un requerimiento con una tecnología nueva para mí, ¿Cuál es mi forma de proceder?
\item Si en mi equipo, después de unos días de acoplamiento no me asignan tarea, ¿Cuál es mi forma de proceder?
\item ¿Cuál ha sido mi experiencia con Protractor y Cucumber?
\item ¿Por qué mi interés de aprenderlos?
\item Preguntaron por algún error que, en mi experiencia, haya cometido y cómo lo afronté
\item ¿Qué aprendí de ese error?
\item Comentaron sobre un posterior uso de AWS y si tengo experiencia ahí.
\item What is inheritance? it's the ability of create classes taking another classes as base, in other words, it allows to derive new classes from other prevoiusly defined classes or interfaces.
\item What is polymorphism? it's the ability to define different behaviors for a method according to its parameters. 
\item What is abstraction? don't need know how certain class or method works to use it. The user of our classes or methods doesn't need to know how they work to use them.
\item What is encapsulation? it's hide every data or function that the user must not manipulate leaving accesible just those properties or methods allowed for that purpose.
\item What is an interface? -OK-
\item What is an abstract class? -OK-
\item Difference between interface and an abstract class? -OK-
\item Singleton -OK-
\item Factory -OK-
\item SOLID: Single, Open-close, Liskov substitution, Interface segregation, Dependency inversion. -OK-
\item What is static? Static declared variables or methods are globally accessible without creating an instance of the class
\item What is serialization? Serialization is the process of converting an object into a form that can be persisted or transported like JSON, XML or binary
\item What is deserialization? It is the process of converting an stream of data into an object
\item What is a reference type? Reference type is a pointer to another memory location where the Object is being stored
\item What is a value type? A value type holds a data value within its own memory space
\item What is boxing? cuando encapsulo un elemento de un tipo específico en un Object.
\item What is unboxing? Cuando extraigo de un elemento Object hacia el tipo específico a fin de procesarlo.
\item What are Generics? 
\item Delegates? Son referencias hacia métodos, esto permite asociar a cierto tipo de variables esos métodos, de esta manera se utilizan en las respuestas de eventos.
\item Which verbs do you know and when do you use those verbs? GET, POST, PUT, DELETE
\item How do you implement PATCH? Used to update/modify elements, if the element exists with the same values then no operation must be performed, else, update the info.
\item What is REST? Representational State Transfer, it means, the data transfer must be uniform, the API must avoid expose too much server details, client \& server must be separated, the request must be stateless.
\item What is SOAP?
\item Can you explain media type formatters?
\item What are WebApi Filters?
\item What is CORS?
\item How do you consume an API from C\#?
\item How can you implement security for an API?
\item What is the difference between ApiController and Controller?
\item What is a microservice?
\item What is a monolith application?
\item aws
\item lambda
\item difference .net and .net core
\item docker deployment
\item pipeline
\item containers
\item jenkins
\item unit tests
\item front end
\item Explain the this keyword: The “this” keyword refers to the object that the function is a property of.The value of “this” keyword will always depend on the object that is invoking the function.
\item Which data types do you know in Javascript: String, number, bigint, null, undefined, symbol
\item What is the difference between undefined and null
\item What is an Immediately Invoked Function in JavaScript? An Immediately Invoked Function ( known as IIFE and pronounced as IIFY) is a function that runs as soon as it is defined.
\item What is a closure Closures is an ability of a function to remember the variables and functions that are declared in its outer scope.
\item New features in ES2015 or ES 6
\end{enumerate}
\end{multicols}
