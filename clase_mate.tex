% !TEX TS-program = pdflatex
% !TEX encoding = UTF-8 Unicode

% This file is a template using the "beamer" package to create slides for a talk or presentation
% - Giving a talk on some subject.
% - The talk is between 15min and 45min long.
% - Style is ornate.

% MODIFIED by Jonathan Kew, 2008-07-06
% The header comments and encoding in this file were modified for inclusion with TeXworks.
% The content is otherwise unchanged from the original distributed with the beamer package.

\documentclass{beamer}

% Copyright 2004 by Till Tantau <tantau@users.sourceforge.net>.
%
% In principle, this file can be redistributed and/or modified under
% the terms of the GNU Public License, version 2.
%
% However, this file is supposed to be a template to be modified
% for your own needs. For this reason, if you use this file as a
% template and not specifically distribute it as part of a another
% package/program, I grant the extra permission to freely copy and
% modify this file as you see fit and even to delete this copyright
% notice. 


\mode<presentation>
{
  \usetheme{Warsaw}
  % or ...

  \setbeamercovered{transparent}
  % or whatever (possibly just delete it)
}


\usepackage[spanish]{babel}
% or whatever

\usepackage[utf8]{inputenc}
% or whatever

\usepackage{times}
\usepackage[T1]{fontenc}
% Or whatever. Note that the encoding and the font should match. If T1
% does not look nice, try deleting the line with the fontenc.


\title[Clase de mate] % (optional, use only with long paper titles)
{Clase de mate}

%\subtitle{Presentation Subtitle} % (optional)

%\author%[Author, Another] % (optional, use only with lots of authors)
%{F.~Author\inst{1} \and S.~Another\inst{2}}
% - Use the \inst{?} command only if the authors have different
%   affiliation.

%\institute[Universities of Somewhere and Elsewhere] % (optional, but mostly needed)
%{
%  \inst{1}%
%  Department of Computer Science\\
%  University of Somewhere
%  \and
%  \inst{2}%
%  Department of Theoretical Philosophy\\
%  University of Elsewhere}
% - Use the \inst command only if there are several affiliations.
% - Keep it simple, no one is interested in your street address.

\date[Short Occasion] % (optional)
%{Date / Occasion}

%\subject{Ecuación de reta pendiente ordenada en el origen}
% This is only inserted into the PDF information catalog. Can be left
% out. 

% If you have a file called "university-logo-filename.xxx", where xxx
% is a graphic format that can be processed by latex or pdflatex,
% resp., then you can add a logo as follows:

% \pgfdeclareimage[height=0.5cm]{university-logo}{university-logo-filename}
% \logo{\pgfuseimage{university-logo}}

% Delete this, if you do not want the table of contents to pop up at
% the beginning of each subsection:
\AtBeginSubsection[]
{
  \begin{frame}<beamer>{Outline}
    \tableofcontents[currentsection,currentsubsection]
  \end{frame}
}


% If you wish to uncover everything in a step-wise fashion, uncomment
% the following command: 

%\beamerdefaultoverlayspecification{<+->}


\begin{document}

\begin{frame}
  \titlepage
\end{frame}

\begin{frame}{Outline}
  \tableofcontents
  % You might wish to add the option [pausesections]
\end{frame}


% Since this a solution template for a generic talk, very little can
% be said about how it should be structured. However, the talk length
% of between 15min and 45min and the theme suggest that you stick to
% the following rules:  

% - Exactly two or three sections (other than the summary).
% - At *most* three subsections per section.
% - Talk about 30s to 2min per frame. So there should be between about
%   15 and 30 frames, all told.

\section{Ecuación de recta con forma pendiente ordenada en el origen}

%\subsection[Short First Subsection Name]{First Subsection Name}

\begin{frame}{Ecuación de la recta.}{Pendiente y ordenada en el origen.}
  % - A title should summarize the slide in an understandable fashion
  %   for anyone how does not follow everything on the slide itself.
\begin{eqnarray}
P = (-2,-1)\\
Q = (0, 2)
\end{eqnarray}
\end{frame}

\begin{frame}{Ecuación de la recta.}{Pendiente y ordenada en el origen.}
  % - A title should summarize the slide in an understandable fashion
  %   for anyone how does not follow everything on the slide itself.
\begin{equation}
\begin{array}{rcl}
P & = & (-2,-1)\\
\\
Q & = & (0, 2)\\
\\
m & = & \dfrac{Y_Q-Y_P}{X_Q-X_P}\\
\\
m & = & \dfrac{2-(-1)}{0-(-2)}=\dfrac{2+1}{0+2}=\dfrac{3}{2}
\end{array}
\end{equation}
\end{frame}

\begin{frame}{Ecuación de la recta.}{Pendiente y ordenada en el origen.}
  % - A title should summarize the slide in an understandable fashion
  %   for anyone how does not follow everything on the slide itself.
\begin{equation}
\begin{array}{rcl}
Q & = & (0, 2)\\
\\
m & = & \dfrac{3}{2}\\
\\
y & = & mx + b \rightarrow (b = Y_Q)\\
\\
y & = & \dfrac{3}{2}x + 2
\end{array}
\end{equation}
\end{frame}

\section{Funciones y ángulos}

\begin{frame}{funciones y ángulos}
\begin{equation}
\begin{array}{rcl}
\theta & \in & Cuadrante_{II}\\
\\
tan(\theta) & = & \dfrac{-4}{3}
\end{array}
\end{equation}
\end{frame}

\begin{frame}{funciones y ángulos}
\begin{equation}
\begin{array}{rcl}
\theta & \in & Cuadrante_{II}\\
\\
tan(\theta) & = & \dfrac{-4}{3}\\
\theta & = & tan^{-1}\left(\dfrac{-4}{3}\right) = -53.1301\\
\\
\theta & = & -53^\circ 7 ' 48.368"
\\
\end{array}
\end{equation}
\end{frame}

\begin{frame}{funciones y ángulos}
\begin{equation}
\begin{array}{rcl}
\theta & = & tan^{-1}\left(\dfrac{-4}{3}\right) = -53.1301\\
\\
\theta & = & -53^\circ 7 ' 48.368"
\\
cos(\theta) & = & 0.6 = \dfrac{6}{10} = \dfrac{3}{5}
\end{array}
\end{equation}
\end{frame}

\section{Premisas y proposiciones}
\begin{frame}
\begin{equation}
\begin{array}{rcl}
Q & \Rightarrow & \neg P \land R\\
Q & : & \text{Llueve}\\
P & : & \text{Podré ir al cine}\\
\neg P & : & \text{NO Podré ir al cine}\\
R & : & \text{Mi novia se enojará}\\

\end{array}
\end{equation}
\end{frame}

\begin{frame}
\begin{equation}
\begin{array}{c|c|c}
\exists \text{: Existe} & \forall \text{: Para todo} & \implies\text{: Implica o entonces}\\
\iff \text{: sí y solo sí} & \lor \text{: O} & \land\text{: Y}\\
\neg\text{: Negación} & \sim\text{: Negación} & \\
\nexists\text{:No existe}
\end{array}
\end{equation}

\begin{equation}
\begin{array}{ccc}
\neg\forall x P(x) & \equiv & \exists x \neg P(x)\\
\neg\exists x P(x) & \equiv & \forall x \neg P(x)\\
\neg(A \land B) & \equiv & \neg A \lor \neg B\\
\neg(A \lor B) & \equiv & \neg A \land \neg B
\end{array}
\end{equation}
\end{frame}

\section{Raíces cuadradas}
\begin{frame}
\begin{equation}
x = 256
\end{equation}
Calcular su raíz cuadrada
\begin{equation}
\begin{array}{l|rcl}
256 & 2 \\
\hline
128 & 2\\
64 & 2\\
32 & 2 & \Rightarrow & 2 \times 2\times 2\times 2\times 2\times 2\times 2\times 2\\
16 & 2 & & 2^2 \times2^2 \times2^2 \times2^2 \\
8 & 2 & & \\
4 & 2& & \sqrt{2^2 \times2^2 \times2^2 \times2^2} = 2\times 2\times 2\times 2\\
2 & 2& & 2\times 2\times 2\times 2 = 16\\
1
\end{array}
\end{equation}
\end{frame}

\begin{frame}
\begin{equation}
x = 144
\end{equation}
Calcular su raíz cuadrada
\begin{equation}
\begin{array}{l|rcl}
144 & 2 \\
\hline
72 & 2\\
36 & 2\\
18 & 2 & \Rightarrow & 2 \times 2\times 2\times 2\times 3\times 3\\
9 & 3 & & 2^2 \times2^2 \times3^2 \\
3 & 3 & & \\
1 & & & \sqrt{2^2 \times2^2 \times3^2 } = 2\times 2\times 3\\
 & & & 2\times 2\times 3 = 12\\

\end{array}
\end{equation}
\end{frame}

\begin{frame}
\begin{equation}
x = 144
\end{equation}
Calcular su raíz cuadrada
\begin{equation}
\begin{array}{rcl}
\sqrt{144} & = & \{18,9\}\rightarrow 13 \rightarrow 12
\end{array}
\end{equation}
\end{frame}

\end{document}


